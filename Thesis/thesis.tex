% options are:
% PhD, MSc (choose one)
% beforeExam (to make the personal thanks invisible)
\documentclass[MSc,beforeExam]{iitcsthesis}
\usepackage{cjhebrew}
\usepackage{amsmath}
\usepackage{amsfonts}
\usepackage{amssymb}
\usepackage{bbm}
\usepackage{graphicx}
\newtheorem{theorem}{Theorem}
\newtheorem{conjecture}{Conjecture}
\usepackage[ruled,vlined,linesnumbered]{algorithm2e}
\usepackage{atbegshi}
\newif\ifRP
\newbox\RPbox
\setbox\RPbox\vbox{\vskip1pt}
\makeatletter
\AtBeginShipout{%
  \ifRP
    \AtBeginShipoutDiscard%
    \global\setbox\RPbox\vbox{\unvbox\RPbox
      \box\AtBeginShipoutBox\kern\c@page sp}%
  \fi
}%
\renewcommand{\RPtrue}{%
  \clearpage
  \ifRP\RPfalse\fi
  \global\let\ifRP\iftrue
}%
\renewcommand{\RPfalse}{%
  \clearpage
  \global\let\ifRP\iffalse
  \setbox\RPbox\vbox{\unvbox\RPbox
    \def\protect{\noexpand\protect\noexpand}%
    \@whilesw\ifdim0pt=\lastskip\fi
      {\c@page\lastkern\unkern\shipout\lastbox}%
  }%
}%
\makeatother


% For some reason, the hebrew package clashes with the amsthm package. If you need a proof environment, you can uncomment the following:
%\usepackage{amssymb} %Needed for \blacksquare. Take care not to add this package twice...
%\newenvironment{proof}[1][Proof]{\par \textbf{#1.} }{\hspace{10pt}\hfill$\blacksquare$\par}
\usepackage{etoolbox}

\makeatletter
\let\heb@everypar\everypar
\def\everypar#1#{\heb@everypar}
\makeatother
\begin{document}

\bibliographystyle{plain}

\authorEnglish{Amit Wolfenfeld}

\titleEnglish{Dueling Bandits Black Boxes}

\supervisorEnglish{The research thesis was done under the supervision
of Prof.~Nir Ailon in the Computer Science Department.}

\GregorianDateEnglish{September 2015}
%\JewishDateEnglish{Tishrei 5769}

%\personalThanksEnglish{Write thanks to professor and others}
%\financialThanksEnglish{The generous financial support of the Technion (and someone else if you got a scholarship) is gratefully acknowledged.}

\maketitleEnglish

\abstractEnglish
In machine learning, the notion of multi-armed bandits refers
to a class of online learning problems, in which a learner (also called decision maker or agent) explores and exploits a given set of choice alternatives in the course of a sequential decision process. 
In the standard setting, the learner learns from stochastic feedback in the form of real-valued rewards.

We study a partial information online learning problem
where actions are restricted to noisy comparisons
between pairs of alternatives - The Dueling Bandits Problem. 
As opposed to conventional approaches
which requires the real valued reward of the chosen arms
to be numerical and observable, the Dueling Bandits setting
only assumes that the (noisy) stochastic feedback about
the relative reward of two chosen arms is readily available.
This type of relative feedback, or pairwise preference, is particularly
fitting in cases where numerical or absolute rewards
are difficult to quantify
(for instance, user preference of a set of search engine
results, preference of taste), but
where pairwise preference are easy to make.

In this paper we propose several new methods for the Dueling Bandit Problem. Our approach extends the Doubler and Sparring algorithm proposed on \cite{ailon2014reducing}. We show empirical results using real data from Microsoft Research's LETOR project.

\abbreviationsAndNotationsEnglish
For the convenience of the reader we have included a table of all the definitions: 

\begin{tabular}{lcl}
			$T$ & --- & Horizon \\
 			$t$ & --- &  Round \\
 			$X$ & --- & Arms space \\
 			$K = |X|$ & --- & Total Number of Arms\\
 			$x_t \in X$ & --- & Left arm \\
 			$y_t \in X$ & --- & Right arm \\
 			$u_t \in \mu_t$ & --- & Left utility \\
 			$v_t \in \mu_t$ & --- & Right utility \\
 			$b_t$ & --- & Observed feedback \\
 			$R_U(T)$ & --- & Total Utility Based Regret till T\\
			$R_P(T)$ & --- & Total Preference Based Regret till T\\
 			$\mathcal{P}$ & --- & Set of potential arms \\
 			$\hat{P}_{x, y}$ & --- & Estimate of $P(x>y)$\\
 			$\hat{C}_{x, y}$ & --- &   Confidence interval of - $(\hat{P}_{x, y} - \sqrt{\log(1/\delta)/t},\hat{P}_{x, y} +\sqrt{\log(1/\delta)/t})$\\
 			$n_x$ & --- & The number of times arm $x$ has been played\\
 			$w_x$ & --- & The number of times arm $x$ has won\\
 			$\hat{P}_x$ & --- &  $ w_x / n_x $\\
 			$ W = [w_{x,y}]$ & --- & Number of wins of arm $x$ over arm $y$\\
 			$G = [g_{x,y}]$ & --- & Utility function\\
 			$\Theta_{x,y}$ & --- & Random variable with Beta distribution.\\
 \end{tabular}
\newpage 
 \begin{tabular}{lcl}
 			MAB & --- & Multi Armed Bandit\\
 			PBDB & --- & Preference Based Dueling Bandits\\
 			UBDB & --- & Utility Based Dueling Bandits\\
 			UCB & --- & Upper Confidence Bound\\
 			IF & --- & Interleaved Filter\\
 			BTM & --- & Beat The Mean\\
 			RUCB & --- & Relative Upper Confidence Bound\\
 			RCS & --- & Relative confidence Sampling\\
 			SAVAGE & --- & Sensitivity Analysis of Variables for the Generic Exploration\\
 			SBM & --- & Singleton Bandits Machine\\
 			IR & --- & Information Retrieval\\
 			LETOR & --- & Learning to Rank
 			
 			
\end{tabular}

\chapter{Introduction}

	Multi-armed bandit (MAB) algorithms have received considerable attention and have been studied quite intensely in machine learning since 1985 when Lai and Robbins released their paper \cite{lai1985asymptotically}. 
	The huge interest in this topic is not very surprising, due to the fact that this MAB setting is not only theoretically challenging but also extremely useful, as can be seen from its use in a wide range of applications. MAB algorithms are used today for solving many problems such as in search engines \cite{radlinski2008learning, agrawal2009diversifying, hofmann2011contextual, slivkins2010ranked}, online advertisement \cite{rusmevichientong2006adaptive, yuan2013adaptive, chakrabarti2009mortal}, and recommendation systems \cite{li2010contextual, li2011unbiased}.

	The multi-armed bandit problem, or bandit problem for short, is one of the simplest instances of the sequential decision making problem, in which a learner needs to select options from a given set of alternatives repeatedly in an online manner. The name comes from the gambling world in which a gambler decides from a row of slot machines (sometimes known as "one-armed bandits") and decides which machines to play, how many times to play each machine, and in which order to play them. When played, each machine provides a random reward from a distribution specific to that machine.
	The objective of the gambler is to maximize the sum of rewards earned through a sequence of lever pulls. 
	To be more precise, the learner selects one option at a time and observes a numerical (and typically stochastic) reward, providing information on the quality of that arm. The goal of the learner is to optimize an evaluation criterion such as the error rate (the expected percentage of playing a suboptimal arm) or the cumulative regret (the expected difference between the sum of the rewards actually obtained and the sum of rewards that could have been obtained by playing the best arm in each round).

	In order to minimize the regret, the learner has to face the crucial tradeoff at each trial between ``exploitation" of the machine that has the apparent highest expected payoff and ``exploration" to get more information about the expected payoffs of the other machines. 
	The learner has to find the best ``ratio" between playing the arms that produced high rewards in the past (exploitation) and trying other, possibly even better arms the (expected) reward of which is not precisely known so far (exploration).

	There are many variations of the MAB problem and in most of them we assume a numerical reward such as ``arm  number 1 has the value of 0.7"; however there are many applications where such assumption does not hold.
In these applications the feedback is a pairwise comparison in the form of
 ``arm number 1 seems better than arm number 2".  % as opposed to standard bandits.

	There are many cases in the world of machine learning where precise feedback is not available, and only preference feedback is available. In these cases %weakly supervised learning and 
preference learning must be used, and this is what we study here.  		 
	to be more precise, in preference learning feedback is typically represented in a purely qualitative way, namely in terms of pairwise comparisons or rankings. 
%	Feedback of this kind can be useful in online learning, too, as has been shown in online information retrieval.

	Web search and Internet marketing are two examples that show the importance for the Dueling Bandits setting. A search engine needs to give the user the best result for his or her query. For every query the search engine lets the user select from several options of search results and receives the feedback according to the users choice. This feedback comes in the form of "the first result is preferred over the other results".  
	Another example is when an advertiser aims to sell products online. The advertiser will direct users to his sale page. 
	Every advertiser wants to sell as much as possible; therefore they will want to improve their sale page. In order to improve their sale page the advertiser will create several versions of the pages, split the users between them, and see which one is the top performer. This process is called A/B testing, and each page version is represented by an arm in the Multi Armed Bandit  setting. 
	The problem with standard bandits is that there are trends in the market that temporarily decrease or increase the overall performance (Christmas time for instance).
	Assuming that the arm's order of performance stays the same, meaning the best arm, performance-wise, stays first, the second best arm stays second and so on - Dueling Bandits Algorithms can be used to increase the advertiser's sale performance while keeping the regret to a minimum.
	Extending the multi-armed bandit setting to the case of preference-based feedback, i.e., the case in which the online learner is allowed to compare arms in a qualitative way, is therefore a promising idea. 
	Indeed, extensions of that kind have received increasing attention in the recent years.
	The aim of this thesis is to provide a survey of the state-of-the-art in the field of Dueling Bandits Algorithms and present several new algorithms. 
	In Chapter 2 we provide a scientific background for the Dueling Bandits Problem. In Chapter 3 we survey the state-of-the-art algorithms. In Chapter 4 we present a new algorithm that out performs the algorithm described in Chapter 3. In Chapter 5 we present %the 
empirical results.


\chapter{Scientific Background}

In this chapter we will go into more detail of what the MAB problem is and more importantly the definition of the Dueling Bandits Problem. 
	We discuss two types of settings, the first - Utility Based Dueling Bandit (UBDB) setting and the second Preference Based Dueling Bandit (PBDB) setting.
	\section{Multi Armed Bandits}
	As described in the previous section, the multi-armed bandit problem is a sequential decision making problem, where a learner explores and exploits a stochastic environment. 
	In this setting, the learner performs actions, referred to as arm pulls.
	The arms belong to an infinite or finite set $X$.
	If the set is finite we denote $|X|$ by $K$.
	Each arm $x \in X$ is associated with a probability distribution over $[0, 1]$, with expectation $\mu_{x}$. 
	Throughout this paper we assume the existence of a unique best arm:
	\begin{equation}\label{eq:regret}
		x^* = argmax_{x \in X}(\mu_{x})
	\end{equation}
	At each round $t > 0$ the learner "pulls" an arm $x_t \in X$ and acquires a stochastic reward or utility $u_t$, independently of all previous rewards (i.i.d). 
	For each arm $x$ and for all rounds $t \geq 1$, $n_{x}(t)$ denotes the number of times arm $x$ has been `'played".
	In this setup the cumulative regret is defined as follows:
	\begin{equation}\label{eq:regret}
		R(T) = \sum_{t=1}^{T} \mu_{x^*} - u_t\ .
	\end{equation}
	The cumulative regret measures the difference between the utility the player could have acquired if he played the best arm and the sum of utilities actually acquired.

\subsection{UCB Algorithm}
	The most commonly studied algorithm for the MAB setting \cite{auer2002finite, auer2010ucb} is known as UCB. %algorithm, is used to minimize the cumulative regret for finite horizon case. 
	It relies on finding a high confidence upper bound for all arms, as seen in line~\ref{line:ref:UCB}, according to the samples so far, called Upper Confidence Bound (hence the name). 
	The bound for arm $x$ decreases as the number of times it had been sampled so far, $n_x$, grows.
	At each round, the sampled arm is the arm that has the highest confidence bound.  Typically, an arm can have a high confidence bound either because it hadn't been sampled enough, or because the arm's average reward has been promising.
	
\noindent
	The UCB algorithm gives the following guarantees:
	
	
\begin{theorem}\label{thm:UCB}[From \cite{auer2002finite}]

		Running the UCB algorithm with $|X|=K$, with a finite time horizon of $ T > K$, the expected regret is bounded by $R(T) = \mathcal{O} \left( \left(\sum_{x\in X \backslash x^*} \frac{1}{\Delta_x} \right)\log T \right)$, where $\Delta_x$ is the gap between the best arm $x^*$ and arm $x$.

\end{theorem}

	%Here we present a variant of UCB that uses an additional parameter $\beta$ to balance between the exploration and exploitation of the algorithm:
	\begin{figure}[h]
	\IncMargin{1em}
		\begin{algorithm}[H]
		\SetKwInput{KwInput}{Input}
		
		\KwInput{$ X,K=|X|$}
			$ t\leftarrow 1$\\
			$\forall x \in X : \hat{\mu_x} \leftarrow 0$\\
			\BlankLine
			
			\While{$t\leq T $}{
				$x_t\leftarrow argmax_{x\in X} \left(\hat{\mu}_x +  \sqrt{\frac{2\ln t}{n_x}}\right)$ \label{line:ref:UCB}\\
 				play arm $x_t$ and acquire $u_t$\\
 				update $\hat{\mu}_x$
			}
			\caption{UCB}
		\end{algorithm}
		\caption{UCB Algorithm}\label{algo_UCB}
	\end{figure}
\newpage
	\subsection{Thompson Sampling Algorithm}
	One of the earliest algorithms, given by W. R. Thompson, dates back to 1933 \cite{thompson1933likelihood}. The basic idea is to choose an arm to play according to its perceived probability of being the best arm,
based on a Beta prior.

	The Thompson Sampling algorithm proceeds as follows.
The original analysis for this algorithm is based on binary feedback, meaning that the utility of an
arm is either $1$ (a ``success'' henceforth) or $0$ (a ``failure'' henceforth).
	

The algorithm maintains the number of successes and failures for each arm, and holds a random variable with $\beta$-distribution for each arm $\Theta_x\sim Beta(Success_x+1,Failures_x+1)$.
	At each round all the random variables $\Theta_{x,t}$ are sampled.
	The chosen arm is then given by $x_t \in argmax_{x\in X} \Theta_{x,t}$.
	While the theoretical behaviour of Thompson sampling has remained elusive for a long time, fairly good understanding of its theoretical properties was achieved by Agrawal and Goyal \cite{agrawal2011analysis, agrawal2012further}, proving the first logarithmic regret bound: 
	
	\begin{theorem}\label{thm:TS}[From \cite{agrawal2011analysis}]

		Running the Thompson Sampling algorithm with $|X|=K$, with a finite time horizon of $ T > K$, the expected regret is bounded by $R(T) = \mathcal{O} \left( \left(\sum_{x\in X \backslash x^*} \frac{1}{\Delta_x^2} \right)^2 \log T \right)$, where $\Delta_x$ is the gap between the best arm $x^*$ and arm $x$.

	\end{theorem}
	The pseudo-code of the Thompson Sampling algorithm can be seen in Figure \ref{algo_TS}.
	
	\begin{figure}[h]
	\IncMargin{1em}
		\begin{algorithm}[H]
		\SetKwInput{KwInput}{Input}
		
		\KwInput{$ X,K=|X|$}
			$ t\leftarrow 1$\\
			$\forall x \in X : Success_{x} \leftarrow 0$\\
			$\forall x \in X : Fails_{x} \leftarrow 0$\\			
			\BlankLine
			
			\While{$t\leq T $}{
				$\forall x \in X : \Theta_{x,t} \sim Beta(Success_{x}+1, Fails_{x}+1)$\\
				$x_t\leftarrow argmax(\Theta_{x,t})$\\
 				play arm $x_t$ and acquire $u_t$\\
 				$Success_{x_t} \leftarrow Success_{x_t} + u_t$\\
 				$Fails_{x_t} \leftarrow Fails_{x_t} + (1-u_t)$
			}
			\caption{Thompson Sampling}
		\end{algorithm}
		\caption{Thompson Sampling Algorithm}\label{algo_TS}
	\end{figure}	
\newpage	
	\section{Dueling Bandits}
		To formalize the problem of learning from preferences, we consider the following interactive online learning model for the K-armed dueling bandit problem \cite{yue2012k, yue2009interactively}.
		At each iteration $t$, the learning system chooses not one but two arms $x_t, y_t \in X$, where as before,
 $X$ is the set (either finite or infinite) of possible actions. The two arms are compared 
by a stochastic process,
and the feedback is returned in the form of a binary random variable $b_t$, encoding which arm beat the other.

There are various ways to define the stochastic comparison process.
		In this thesis we will study two different  settings.
	
	In the first, Utility Based Dueling Bandits \textbf{UBDB}, each arm $x\in X$ is  (just as in standard MAB) equipped
with a latent expected  utility.   When two arms $x_t,y_t$ are played,  just as in MAB random utilities are drawn independently
for each arm.  However, unlike MAB these utilities are \emph{not} viewed. As stated above, only a noisy
comparison is viewed.
The noisy compaison ``respects'' the drawn utilities in the sense
that the arm with higher utility has a better chance of beating the other.  Pictorially, this can
be shown as a match between two boxers both of whom have a latent expected skill level.  The actual performance 
of each boxer during the match is a random variable distributed around his corresponding expectation, and the match
outcome is a coin flip biased toward the boxer with the higher performance.
	
	In the second - \textbf{PBDB} -- Preference Based Dueling Bandits, the outcome of the comparison between any two  arms behaves like a biased coin flip, where the bias depends on the identity of the two arms only.  The latent matrix defining these random coins, which we denote by
$P$, does not necessarily
have any structure, although some of the algorithms we study below for PBDB will assume some
social choice theoretical properties that will be explained below.   Note that in PBDB there
is no underlying utility for each arm.  This makes the definition of regret challenging.
We will borrow some standard definitions from the literature.

It is important to  note that UBDB is a special case
of PBDB in which the matrix $P$ is induced by a value function assigned to the different arms.
Although in a sense UBDB is the simplest form of dueling bandits, we will show below that the theory
behind it is nevertheless rich.

%
%The matrix of  behaves according to latent general  preference matrix $P$ that characterizes the relationship between the arms.
	\subsection{UBDB}
At iteration $t$, let  $u_t$ and $v_t$ be the drawn (hidden) utilities of the chosen arms $x_t$ and $y_t$,
respectively.
%		The quantity $u_t$ (resp. $v_t$) is drawn from a distribution that depends on $x_t$ (resp. $y_t$) only.
		We assume, as always, these utilities are in $[0, 1]$.
		The learner is rewarded with the average utility $$U_{av}(t) = (u_t + v_t)/2$$ %
%\footnote{Where \textit{av} is shorthand for average} 
of the two actions it presents, but it does not observe this reward. 
		 As explained above, instead, it only observes a binary choice among the two alternative arms $x_t, y_t$, which depends on the respective utilities $u_t$ and $v_t$.
		In particular, we model the observed choice as a binary random variable $b_t$, which takes the
value $1$ if $x_t$ (of utility $u_t$) wins, otherwise $0$.
 A link function $\phi :[0, 1] \times [0, 1] \rightarrow [0, 1]$ posits the
connection between the two utilities and the distribution of $b_t$, as follows:
		\begin{equation}\label{eq:b_t}
		\begin{cases}
		P(b_t = 1|u_t, v_t) = \phi(u_t, v_t)
		\\
		P(b_t = 0|u_t, v_t) = \phi(v_t, u_t)\ .
		\end{cases}
		\end{equation}
%	where $\phi:[0, 1] \times [0, 1] \rightarrow [0, 1]$ is a link function.
	Clearly, the link function has to satisfy $\phi(A, B) + \phi(B, A) = 1$.
	Henceforth we will use the notation $x_t \succ y_t $ for the even `'$x_t$ beats $y_t$ in round $t$".
%	We assume $u_t$ is drawn from a distribution of expectation $\mu(x_t)$ and $v_t$ independently from a distribution of expectation $\mu(y_t)$.
	The link function $\phi$, which is assumed to be known, quantitatively determines how to translate the utilities $u_t, v_t$ to winning probabilities.
	The linear link function $\phi_{lin}$ is defined by
	\begin{equation}\label{eq:link_function}
		P(b_t = 1|u_t, v_t) = \phi_{lin}(u_t, v_t) = \frac{1+v_t-u_t}{2}\in [0,1]\ .
	\end{equation}

	For the UBDB case the definition of the regret is very natural and straightforward:

	\begin{equation}\label{eq:utility_regret}
		R_U(T) = \sum_{t=1}^T \mu_{x^*}-U_{av}(t)\ .
	\end{equation}
	Where, as always, $x^* = argmax_{x \in X} \mu_{x}$.
	This implies that the expected zero regret is achievable by setting $(x_t, y_t) = (x^*,x^*)$.
	It should be also clear that playing $(x^*,x^*)$ is pure exploitation, because the feedback is then an unbiased coin with zero exploratory information.
	
	\subsection{PBDB}
		Consider a fixed set of arms $X = \{x_1,..,x_k\}$.
		As actions, the learner performs a comparison between any pair of arms $x_t$ and $y_t$, meaning the action space is identified with the set of index pairs $(x,y)\in X\times X$. %such that $1 \leq i \leq j \leq K$. 
		In this work  we characterize the feedback of the comparison by an unknown preference matrix $P$, which is not necessarily devised from a latent utility.
		More precisely
		
		\begin{equation}\label{eq:preference_matrix}
			P = [p_{x,y}] \in [0,1]^{K \times K} \ .
		\end{equation}
		To be more precise - for each pair of arms $(x ,y)$, this relation specifies the probability of the event
		\begin{equation}\label{eq:p_xy}
			Pr(x \succeq y) = p_{x, y}
		\end{equation}
		of observing a preference for $x$ in a direct comparison with $y$.
		Meaning, each $p_{x,y}$ defines a Bernoulli distribution.
		
%Throughout this paper we assume these probabilities are independent and stationary during all rounds $t > 0$.
		When  two arms $(x_t ,y_t)$ are played and compared at time $t$, the outcome bit $b_t$ is distributed as an independent Bernoulli , without any dependencies on the previous iterations.
		The relation matrix $P$ is reciprocal in the sense that $p_{x,y} = 1-p_{x,y}$ for all $x,y \in X$.
		Arm $x$ is said to outperform arm $y$ if $p_{x,y} > 1/2$, meaning the probability of winning in a pairwise comparison is larger for $x$ than it is for $y$. 
		The closer $p_{x,y}$ is to $1/2$, the harder it is to distinguish between arm $x$ and arm $y$ based on a finite sample set from $Pr(x \succeq y)$. This resembles the case in the standard MAB problem where the gap $\Delta_{x,y}$ is very small.
		% When $p_{x,y} = 1/2$, the learner cannot decide which arm is better based on a finite number of pairwise comparisons.

		 		 Defining a regret is more tricky in PBDB. In particular, it is necessary to make assumptions on $P$ for the definition to make sense.
		 In \cite{yue2011beat} "Relaxed Stochastic Transitivity" is assumed, defined by:
		 For any triplet of arms $x \succ y \succ z$ (with respect to some latent underlying order) and some $ \gamma \geq 1 $,
		 we assume $$\gamma \Delta_{x,z} \geq \max(\Delta_{x,y},\Delta_{y,z})\ .$$
		 Where $\Delta_{x,y}$ is defined
		 \begin{equation}\label{eq:delta}
		 	\Delta_{x,y} = p_{x,y} - 1/2
		 \end{equation}
		 In a later paper \cite{urvoy2013generic} this assumption was relaxed, and only a Condorcet winner is assumed, where a Condercet winner is defined as an arm $x$, such that $ \forall y, p_{x,y} > 1/2$. 
		 Given the existence of a Condorcet winner which we denote $x^*$, we define regret for each time-step as follows \cite{yue2011beat}: if arms $x$ and $y$ were chosen for comparison at time $t$, then the  regret at that time is $ \frac{\Delta_{x^*,x_t}+\Delta_{x^*,y_t}}{2} $ and the cumulative regret is
		 \begin{equation}\label{eq:preference_regret}
		 R_P(T) = \sum_{t=1}^T \frac{\Delta_{x*,x_t}+\Delta_{x*,y_t}}{2}\ .
		 \end{equation}

%		 As opposed to the standard bandit game $\Delta_{x,y}$ can be negative, in which the quantity used for the multi-armed bandit task is always positive and depends on the gap between the means of the best arm and the suboptimal arms.
		 \subsection{The Relation Between Preference and Utility based Regrets}
	
	In the extreme case where the preference matrix $P$ is induced by \textbf{UBDB}, we argue that the regret defined in section 2.2.1 (using the linear link function) is the same as in section 2.2.2.
More precisely, we will show that using the definition of the linear link function both utility based regret and preference based regret are the same (up to a factor of 2):
	\begin{equation}\label{eq:ref1}
		p_{x^*,y} = \phi_{lin}(\mu_{x^*},\mu_{y}) = \frac{1 +\mu_{y}-\mu_{x^*}}{2}
	\end{equation}
	Incorporating \eqref{eq:ref1} in the definition of $\Delta_{x,y}$ we get
	$$
	\Delta_{x^*,y} = p_{x^*,y} - \frac{1}{2} = \frac{\mu_{x^*}-\mu_{y}}{2}
	$$
	And so the total regret is defined:
	\
	$$ R_P(T) = \sum_{t=1}^T \frac{\Delta_{x^*,x_t}+\Delta_{x^*,y_t}}{2} = 
\sum_{t=1}^T \frac{\frac{\mu_{x^*}-\mu_{x_t}}{2}+\frac{\mu_{x^*}-\mu_{y_t}}{2}}{2} =
$$

$$
\frac{1}{2} \sum_{t=1}^T \mu_{x^*} -\frac{
	\mu_{x_t}+\mu_{y_t}}{2} =
\frac{1}{2} \sum_{t=1}^T \mu_{x^*}- U_{av} = \frac{1}{2}R_U(T)\ .$$

	\section{Probability Estimation}
		The Dueling Bandit game is played in discrete rounds, either through a finite time horizon or an infinite horizon. 
		 As described in the previous section, the learner compares between two arms in each round $t > 0$. 
		 And so, in each round $t$, the learner selects a pair of arms $x_t,y_t$ and observes
	\begin{equation}
		\begin{cases}
    		x_t \succeq y_t & \text{with probablity } p_{x_t,y_t} 
       	\\
    		y_t \succeq x_t & \text{with probablity } p_{y_t,x_t}
	\end{cases}
	\end{equation}
	In this thesis the pairwise probabilities $p_{x,y}$ can be estimated according to the finite sample sets.
	We consider the set of rounds among the first $t$ iterations, in which the learner decides to compare arms $x$ and $y$, and denote the size of this set by $n_{x,y}$, or the number of times $x$ and $y$ have been compared.
	We denote the number of times $x$ "beat" over $y$ by $w_{x,y}$ and $w_{y,x}$ the number of times $y$ "beat" $x$.
	It is easy to see that $n_{x,y} = n_{y,x} = w_{x,y}+w_{y,x}$ and so the unbiased estimation of $p_{x,y}$ up to iteration $t$ is then given by
	\begin{equation}\label{eq:est_pxy}
		\hat{p}_{x, y} = \frac{w_{x,y}}{n_{x,y}} = 
		\frac{w_{x,y}}{w_{x,y}+w_{y,x}}
	\end{equation}
	As mentioned above, in this thesis we assume that the samples are independent and identically distributed.
	The quantity  $\hat{p}_{i,j}$  is an inbiased estimator of $p_{x,y}$.
	As in most MAB algorithms a high probability confidence interval is obtained by the Hoeffding bound which we
	remind the reader in Theorem~\ref{thm:Hoeffding} below.
	The confidence intervals may differ from one algorithm to another, but usually they are of the form $$[\hat p_{x,y}- c(x,y), \hat p_{x,y}+c_{x,y} ]\ .$$
	We will henceforth say that $x$ outperforms arm $y$ with high likelihood if the upper end of the corresponding
	confidence interval is above $1/2$, namely if  $p_{x,y} + c_{x,y} > 1/2$.  Similarly, we say that  $x$ is beaten by arm $y$ with high confidence if , if $p_{x,y} + c_{x,y} < 1/2$.
	\subsection{Hoeffding Inequality}
	As a preliminary step, we review Hoeffding's inequality.
	\begin{theorem}\cite{Hoeffding}\label{thm:Hoeffding}
	Suppose $\{X_1,..,X_N \}$ are independent random variables with values in the interval $[a,b]$.
	We denote $\bar{X}_N = \sum_{i=1}^N \frac{X_i}{N}$.\\
	If $E[\bar{X}_N] = \mu_X$, then for any $d>0$:
	\begin{subequations}
	\begin{equation}
	Pr(\bar{X}_N \geq \mu_X+ d) \leq e^{-\frac{2d^2}{(b-a)^2}\cdot\frac{1}{N}}
	\end{equation}
	\begin{equation}
	Pr(\bar{X}_N \leq \mu_X- d) \leq e^{-\frac{2d^2}{(b-a)^2}\cdot\frac{1}{N}}
	\end{equation}
	\end{subequations}
	\end{theorem}

\chapter{Survey of Algorithms for Dueling Bandits}
	We start by surveying dueling bandit algorithms. In this section we have included Interleaved Filtering \cite{yue2012k}, Beat the Mean Bandit \cite{yue2011beat}, RUCB \cite{zoghi2013relative}, RCS \cite{zoghi2014relative}, SAVAGE \cite{urvoy2013generic} and the Sparring and Doubler algorithms \cite{ailon2014reducing} for they served as a foundation for our new approach. We categorize these algorithm based on the following criteria - Explore then Exploit algorithms and Pure Exploitation algorithms.

\section{Explore then  Exploit Algorithms}
	In the finite horizon case, most PBDB algorithms are based on the idea of splitting the rounds into two phases, exploration and then exploitation. In the first phase the algorithm identifies the best arm with high probability. In the second phase the algorithm repeatedly compares the chosen arm to itself.
%	Such algorithms, using the above principle are called "explore-then-exploit" algorithms.
	
	The main drawback for these algorithms is that $T$, the horizon, needs to be known in advance. The horizon is needed for the algorithm to control the trade-off between exploration and exploitation and to control the regret accumulated in the event of failure to identify the best arm in the exploration phase.
	To be more specific, let's assume the algorithm can identify the best arm with probability of at least $1-\delta$. We will set $\delta$ to $1/T$ and so the algorithm can guarantee that the best arm is chosen within $T$ rounds with probability greater than $1-1/T$.
	Let's define $R_{explore}(T),R_{exploit}(T)$ as the regrets for the exploration phase and exploitation phase that define the total regret:
	$$R(T) = R_{explore}(T)+R_{exploit}(T)\ .$$
	Assuming the algorithm made a mistake and did not choose the best arm at the end of exploration phase, the regret accumulated, $R_{explore}(T)+R_{exploit}(T)$ is at most $\mathcal{O}(T)$, since the regret accumulated in each round is bounded by 1.
	Assuming the algorithm did choose the best arm, $R_{exploit}(T)$ is identically $0$.
	Consequently, the expected regret of an explore-then-exploit algorithm is
	\begin{equation}
		E[R(T)] \leq (1-1/T)E[R_{explore}(T)] +(1/T)\mathcal{O}(T) = \mathcal{O}(E[R_{explore}(T)]+1)
	\end{equation}
	The same argument holds for the case of high probability regret bounds in the explore-then-exploit framework.
	In summary, the performance of an explore-then-exploit algorithm is bounded by the performance of the exploration algorithm.
	Although Explore and Exploit algorithms need to know the Horizon $T$ in advance, by using a simple workaround, the "Squaring Trick", these algorithm can be used in the infinite horizon setting. An estimated horizon $\hat{T}$ is submitted to the algorithm and once the algorithm reaches $t=\hat{T}$ the horizon is set to $\hat{T}^2$ and the algorithm restarts. This process continues until the algorithm terminates, and hence the name "Squaring Trick".
	
	\section{Pure Exploration Algorithms}
	In the infinite time horizon case no horizon is specified and the game  continues indefinitely.
%	In this case it is not sufficient for the algorithm to maximize accuracy or minimize regret after a single horizon is reached. 
	The algorithm  must minimize regret by constantly  decreasing the frequency of comparisons involving suboptimal arms. %particularly those that fare worse in comparison to the best arm.
%	In the standard bandit problems, the arm that is chosen in each round produces an expected reward that is at most $\epsilon$ shy of the optimal arm, the $\epsilon$-optimal arm.
%	In the PBDB case approximation errors are harder to define.
	The algorithm continues until certain stopping condition is satisfied unlike Explore then Exploit algorithms that terminate after a predefined horizon $T$. 
	%An algorithm is an ($\epsilon, \delta$)-PAC PBDB algorithm with a sample complexity $B(P,\epsilon,\delta)$, if it returns the $\epsilon$-optimal arm with probability at least $1-\delta$ , and the number of comparisons taken by the algorithm is at most $B(P,\epsilon,\delta)$.
	%The sample complexity of the learner is then the number of pairwise comparisons it makes before concluding the optimal arm, and the corresponding bound is denoted $B(P,\epsilon,\delta)$ where $P$ is defined at \eqref{eq:preference_matrix}.

	\section{The Algorithms}
\subsection{Interleaved Filter}
	Yue et al. \cite{yue2012k} propose an algorithm for PBDB in  a finite horizon setting.
	In their setting, they assume existence of an underlying latent complete linear ordering
	over the set $X$.  Note that this ordering is not induced by utilities, and is completely
	nonparameteric.  We denote this ordering using $\succ$.
	
	Their algorithm works for the scenario in which the preference matrix satisfies the following properties:

	\begin{enumerate}

		\item Strong stochastic transitivity.

		\item Stochastic triangle inequality.

	\end{enumerate}
	Strong stochastic transitivity means that for any triplet of arms $x \succ y \succ z$ we assume $\Delta_{x,z} \geq \max(\Delta_{x,y},\Delta_{y,z})$.
	Stochastic triangle inequality means that for any triplet of arms $x \succ y \succ z$ we assume $\Delta_{x,z} \leq \Delta_{x,y} + \Delta_{y,z}$.
	The regret is defined as in \eqref{eq:preference_regret}.
	The algorithm implements an explore-then-exploit strategy.
	This means that the time steps $1..T$ are divided into two phases, the first serving for exploration only (learning) and the second for exploitation only.
	In particular, in the exploitation phase, the algorithm plays only the apparent optimal action consistently.
	The exploration phase works by repeated elimination as follows.
	At each step the algorithm maintains a working set of arms that are potentially optimal.
	It is shown that with high probability the true optimal arm is consistently in the working set.
	The elimination is done using a round robin tournament strategy, as shown in line~\ref{lst:line:rr} of algorithm~\ref{algo_IF}.
	Each arm in the working set is compared against all the arms in the working set, until sufficient confidence is achieved to determine whether the arm must be eliminated (removed from the working set).
%	By sufficient confidence we mean that with high probability an arm cannot be the optimal. 
This is encoded in the following condition (line~\ref{lst:line:elimination}):
	 $$\hat p_{x,y}+c_{x,y}<1/2\ .$$
	 
	Apart from the round robin elimination, a pruning process is executed, as we now explain.
%	We explain the pruning process more precisely.
	As can be seen in line~\ref{lst:line:pruning}, an element $y$ is removed from the working set $\cal P$ if $$\hat p_{x,y}-c_{x,y}<1/2$$ 
	In words, this condition implies with high confidence that $y$ is inferior compared to the current candidate in the round robin.
	Once a single arm is left in the working set, the algorithm enters the exploitation phase.
	The algorithm pseudo-code is presented in Figure~\ref{algo_IF}. 
	The guarantee provided is as follows:

	\begin{theorem}\label{thm:IF}[From \cite{yue2012k}]

		Running the Interleaved Filter algorithm with $|X|=K$, with a known finite time horizon of $T>K$, the expected regret is bounded by $R_P(T) = \mathcal{O} \left( \frac{K\log K}{\Delta_{x^*,y}} \log T \right)$, where $y$ is an arm that is only beaten by $x^*$.

	\end{theorem}
	From the analysis of the algorithm it can be seen that the algorithm returns the best arm with probability of $1-\frac{1}{T}$.
	Correspondingly, a suboptimal arm is returned by the algorithm with probability of $\frac{1}{T}$ hence , similar to a PAC type argument: With high probability of at least $1-\frac{1}{T}$, the estimated regret after $T$ steps is at most $\mathcal{O} \left( \frac{K\log K}{\Delta^*} \log T \right)$. The regret is defined as in \eqref{eq:preference_regret}.
	
	\begin{figure}[h]
	\IncMargin{1em}
		\begin{algorithm}[H]
		\SetKwInput{KwInput}{Input}
		
		\KwInput{$T$,$ X,K=|X|, \delta$}
			$ t\leftarrow 1$\\
			$ \mathcal{P} \leftarrow X$
			Choose $x_t \in \mathcal{P}$ randomly\\
			$ \mathcal{P} \leftarrow \mathcal{P} \backslash x_t$\\
			\BlankLine
			\While{$|\mathcal{P}|>1 \wedge t\leq T$}{

 				\For{$x \in \mathcal{P}$\label{lst:line:rr}}{
					compare($y, x_t$)\\
					update($\hat{p}_{x_t, y}$)\\
					$t \leftarrow t+1$	
				}
				$c_t \leftarrow \sqrt{log(1/\delta)/t}$\\
				\While{$\exists y \in$ s.t. $\left( \hat{p}_{x_t,y}+c_t<1/2 \right)$\label{lst:line:elimination}}{
				$\mathcal{P} \leftarrow \mathcal{P} \backslash \{y\}$
				}
				\If{$\exists y \in$ s.t. $( \hat{p}_{x_t,y}-c_t <1/2 )$\label{lst:line:pruning}}{
					$x_{t+1} \leftarrow y$\\
					$\mathcal{P} \leftarrow \mathcal{P} \backslash \{ y \}$\\
					$\forall x \in \mathcal{P}$ reset $\hat{p}_{y,x}$
				}
			}
			\Return{$x_t \in \cal P$}
			\caption{Interleaved Filter}
		\end{algorithm}
		\caption{Interleaved Filter Algorithm}\label{algo_IF}
	\end{figure}
	
\newpage
\subsection{Beat The Mean Bandit}
	Yue and Joachims \cite{yue2011beat} proposed an explore-then-exploit strategy for a finite horizon setting, very similar to IF. 
	Beat the Mean (BTM) is a preference-based online learning algorithm.
	This algorithm is based on relaxed assumptions with regards to the IF algorithm:
	\begin{enumerate}
		\item Relaxed stochastic transitivity.
		\item Stochastic triangle inequality
	\end{enumerate}
	Relaxed stochastic transitivity means that for any triplet of arms $x \succ y \succ z$ and for some $\gamma \geq 1$ we assume $\gamma\Delta_{x,z} \geq \max(\Delta_{x,y},\Delta_{y,z})$.
	The main idea of BTM is to maintain a score of the "Mean Bandit" (arm in our case) and eliminating arms that are inferior to it with enough confidence.
	Similar to the IF algorithm, BTM maintains a working set of potentially optimal arms.
	At each round the algorithm picks an arm that has the least plays in the corresponding current history, $n_x$, and compares it with a randomly chosen arm from the working set.
	Comparing an arm against random arms in the working set is functionally identical to comparing an arm against the "Mean Bandit" sampled uniformly from the working set.
	The arm, $x$, that is empirically the worst arm and is separated enough from the best arm with enough confidence is defined using:
	\begin{equation}\label{eq:BTM_el}
	min_{x\in \mathcal{P}} (\hat{P}_{x})+c^* \leq max_{y \in \mathcal{P}}(\hat{P}_{y})-c^*
\end{equation}	 
	Once an arm reaches \eqref{eq:BTM_el} (see line~\ref{alg:BTM:elimination}) it is eliminated from the working set. The confidence is defined as $ c^* = c_{\delta, \gamma}(n^*) $, where $c_{\delta, \gamma}(n) = 3\gamma^2 \sqrt{\frac{1}{n}\log \frac{1}{\delta}}$.
	The algorithm pseudo-code is presented in Figure~\ref{algo_BTM}.	
	The guarantee provided is as follows:
	\begin{theorem}\label{thm:BTM}[From \cite{yue2011beat}]
		Running the Beat the Mean Bandit algorithm with $|X|=K$, with a known finite time horizon of $T>K$, the expected regret is bounded by $R_P(T) = \mathcal{O} \left( \frac{\gamma^7K}{\Delta_{x^*,y}} \log T \right)$, where $y$ is an arm that is only beaten by $x^*$.

	\end{theorem}
	
	Similarly to the analysis of the IF algorithm, it can be seen that the algorithm returns the best arm with probability of $1-\frac{1}{T}$.	
	
	\begin{figure}[h]
	\IncMargin{1em}
		\begin{algorithm}[H]
		\SetKwInput{KwInput}{Input}
		
		\KwInput{$T$,$X,K=|X|, c_{\delta, \gamma}(\cdot)$}
			$ \mathcal{P} \leftarrow X $\\
			$ W = [w_{x,y}]$\\
			$ \forall x \in \mathcal{P} , n_x \leftarrow 0 $\\
			$ \forall x \in \mathcal{P} , w_x \leftarrow 0 $\\
			$ \forall x \in \mathcal{P} ,\hat{P}_x \leftarrow w_x / n_x$ or $0$ if $n_x = 0 $\\
			$t \leftarrow 0$ \\
			\BlankLine
			\While{$|\mathcal{P}|>1\wedge t\leq T$}{
				$x_t \leftarrow argmin_{x \in \mathcal{P}} n_x$\\
				select $y_t\in \mathcal{P}$ at random\\
				compare($x_t$, $y_t$)\\
				update($W$)\\
				$n_{x_t} \leftarrow n_{x_t}+1$\\ 
				$ n^*  \leftarrow \min_{x \in \mathcal{P}} n_x$\\
				$ c^* \leftarrow c_{\delta, \gamma}(n^*) $ or $1$ if $ n^* = 0$\\
				\If{$min_{y\in \mathcal{P}} (\hat{P}_{y})+c^* \leq max_{x \in \mathcal{P}}(\hat{P}_{x})-c^*$ \label{alg:BTM:elimination}}{
					$y\leftarrow argmin_{x\in \mathcal{P}}\hat{P}_{x}$\\
					$\forall x \in \mathcal{P}$ delete comparison with $y$ from $w_y, n_y$\\
					$\mathcal{P} \leftarrow \mathcal{P}\backslash\{y\} $\\
				} 
				$t \leftarrow t+1$
			}
			\Return{$x_t \in \cal P$}
			\caption{Beat the Mean Bandit}
		\end{algorithm}
		\caption{Beat the Mean Bandit Algorithm}\label{algo_BTM}
	\end{figure}
\newpage
	\subsection{RUCB} Relative Upper Confidence Bound by Zoghi et al. \cite{zoghi2013relative}, adapts the most commonly used algorithm UCB [Algorithm~ \ref{algo_UCB}] in the standard MAB setting to the Dueling Bandit setting (PBDB version).
	The algorithm assumes that a Condorcet winner exists.
	A Condorcet winner is a definition borrowed from social choice \cite{black1948rationale}.  Roughly speaking, a Condorcet winner beats all the other arms on average. Mathematically, a Condorcet winner (if exists) is defined by: $x$ is the Condorcet winner if $ p_{x,y} > 1/2, \forall y $.
	The algorithm pseudo-code is presented in Figure~\ref{algo_RUCB}.
	Similarly, the UCB algorithm RUCB maintains an upper confidence bound $g_{x_t,y_t}$ for each pair of arms, as seen in line~\ref{alg:rucb:line:ucb_set} of Algorithm~\ref{algo_RUCB}.
	On each round, RUCB selects the arms to compare from the set of potential Condorcet winners, meaning the set of arms for which all are potentially Condorcet winners with sufficiently high probability. 
	After selecting the set of potential Condercet winners the algorithm selects the pair of arms with the highest upper confidence bound $g_{x_t,y_t}$; this can be seen as a Dueling Bandit version of the `'optimism in the face of uncertainty" principle from UCB.
	Note that in line~\ref{alg:rucb:line:ucb_choose_set} the comparisons are based on $g_{x_t,y_t}$ and in line~\ref{alg:rucb:line:ucb_choose} of Algorithm~\ref{algo_RUCB} the comparisons are based on $g_{y_t,x_t}$, making it more difficult for the algorithm to compare an arm against itself.
	Thus, RUCB strives to avoid auto-comparisons (comparisons of an item with itself) until there is great certainty that $x_t$ is indeed the Condorcet winner.
	Roughly speaking, if the algorithm would pick the arm $y_t$ according to $g_{x_t,y_t}$ and not $g_{y_t,x_t}$, the algorithm will pick the arm according to the impact of $x_t$ and not $y_t$ and is more likely to pick the pair $x_t,y_t = x_t,x_t$.
	The RUCB algorithm guarantees the following:
	\begin{theorem}\label{thm:RUCB}[From \cite{zoghi2013relative}]

		Running the RUCB algorithm with $|X|=K$, with a finite time horizon of $T>K$, the expected regret is bounded by $R_P(T) = \mathcal{O} \left( \frac{K log T}{\Delta_{x^*,y}} \right)$, where $y$ is an arm that is only beaten by $x^*$.

	\end{theorem}
	
	\begin{figure}[h]
	\IncMargin{1em}
		\begin{algorithm}[H]
		\SetKwInput{KwInput}{Input}
		
		\KwInput{$T$,$X,K=|X|, \alpha >1/2,T \in \{1,2,... \}$}
		$ W = [w_{x,y}]$\\
		\BlankLine
		\While{$t \leq T $}{
			$G = [g_{x,y}] \leftarrow \frac{w_{x,y}}{w_{x,y}+w_{y,x}} +\sqrt{\frac{\alpha \cdot \ln(t)}{w_{x,y}+w_{y,x}}}$ \label{alg:rucb:line:ucb_set}\\
			$\forall x \in X : u_{x,x} \leftarrow 0 $\\
			Pick any $x_t$ that satisfies $g_{x_t,x} \geq 1/2, \forall x$. \label{alg:rucb:line:ucb_choose_set}\\
			If no $x_t$ exists pick $x_t$ randomly from $X$.\\
			$y_t \leftarrow argmax_{y} g_{y,x_t}$\label{alg:rucb:line:ucb_choose}\\
			compare($x_t$ , $y_t$)\\
			update($W$)\\
			$t\leftarrow t+1$
} 
			\caption{RUCB}
		\end{algorithm}
		\caption{RUCB Algorithm}\label{algo_RUCB}
	\end{figure}
	
	\newpage
\subsection{RCS}
	(Munos et al., 2014) Relative Confidence Sampling \cite{zoghi2014relative} is very similar to RUCB in the sense that it aims to minimize cumulative regret and does not eliminate arms from a working set.
	The algorithm pseudo-code is presented in Figure~\ref{algo_RCS}.
	Similarly to the RUCB algorithm, the RCS algorithm's only assumption is the existence of a Condorcet winner.
	
	On each round the algorithm simulates a tournament between all the arms, against all the arms.
	The simulation is conducted, as seen in line~\ref{alg:rcs:simulation}, according to a random variable $\Theta_{x,y} $ with $Beta(\alpha,\beta)$ distribution. 
	The hyper-parameters, $\alpha,\beta$, are $w_{x,y}+1,w_{y,x}+1$, respectively, where $w_{x,y}$ is the number of times arm $x$ beat arm $y$.
	The main idea behind this simulation technique is to use the superior performance of Thompson Sampling \cite{agrawal2011analysis} in the same manner it is used in the K-armed bandit setting. 
	Once the simulation is over, similarly to the RUCB algorithm, the RCS algorithm select a set of potential Condorcet winners, as seen in line~\ref{alg:rcs:select_x}, and by this the algorithm selects arm $x_t$.
	In line~\ref{alg:rcs:select_y} the algorithm selects the arm $y_t$ according to the "optimism in the face of uncertainty" principle, the same as the UCB algorithm.
	After both arms are selected and compared, the algorithm updates $w_{x,y}$ for all the arms and continues to the next round.
	Although no theoretical guarantees were presented in \cite{zoghi2014relative}, according to the experiments there, RCS outperforms the RUCB algorithm for a small number of arms (less than 40).	
	In Figure~\ref{alg:RCS:exp} we can see the results of the experiments Munos et al. performed on the MSLR-WEB30K dataset.
	To evaluate the RCS algorithm, Munos et al. applied it to the problem of scorer (which was inferred as ranker in the original paper) evaluation from the field of information retrieval (IR). 
	A ranker is a function that takes a user's search query and ranks the documents in a collection according to their relevance to that query. 
	Ranker evaluation aims to determine which set of rankers performs best.
	Given a set of K rankers, the problem of finding the best ranker can then be modelled  as a K-armed Dueling Bandit problem, with each arm relating to a ranker.
	These experiments are built on real IR data from the MSLR-WEB30K dataset.
	We will go into more detail on the experiments in chapter 5.
	Using this data set, 20, 30, and 40 rankers each relate to a ranking feature provided in the data set, e.g., PageRank.
	The ranker evaluation task then continues to conclude which single feature is the best ranker (Hofmann et al., 2013)\cite{hofmann2013fidelity}.
	
	\begin{figure}[h]
	\includegraphics[scale=0.2]{graphs/RCS.jpg}
	\caption{Cumulative regret averaged over 90 runs}\label{alg:RCS:exp}
	\end{figure}
	\begin{figure}[h]
	\IncMargin{1em}
		\begin{algorithm}[H]
		\SetKwInput{KwInput}{Input}
		
		\KwInput{$T$,$X,K=|X|, \alpha >1/2,T \in \{1,2,... \}$}
		$ W = [w_{x,y}]$\\
		\BlankLine
		\While{$t \leq T$}{
	
	\emph{assume an arbitrary injective mapping index$:X\rightarrow[K]$, \\
	inducing an arbitrary canonical order on $X$.}\\
	\For{$x,y \in X$ s.t. $index(x)<index(y)$ \label{alg:rcs:simulation}}{
		 $\Theta_{x,y}(t) \sim Beta(w_{x,y}+1,w_{y,x}+1 )$\\
		 $\Theta_{y,x}(t) = 1-\Theta_{x,y}(t)$
		 }
	Pick any $x_t$ that satisfies $\Theta_{x_t,x} \geq 1/2, \forall x$.\label{alg:rcs:select_x}\\ 
	If no $x_t$ exists pick $x_t$ that was chosen least often ($argmin_x n_x)$.\\
	$G = [g_{x,y}]\leftarrow \frac{w_{x,y}}{w_{x,y}+w_{y,x}} +\sqrt{\frac{\alpha \cdot \ln(t)}{w_{x,y}+w_{y,x}}}$ \\
	$G_{z,z} \leftarrow 0 $ for each $z \in X$\\
	$y_t \leftarrow argmax_{y} g_{y,x_t}$\label{alg:rcs:select_y}\\
	compare($x_t$, $y_t$)\\
	update($W$)\\
	$t\leftarrow t+1 $
} 
			\caption{RCS}
		\end{algorithm}
		\caption{RCS Algorithm}\label{algo_RCS}
	\end{figure}
	
	\newpage
	
	
	
\subsection{SAVAGE}
	(Uryoy et al., 2013)\cite{urvoy2013generic} Sensitivity Analysis of Variables for the Generic Exploration (SAVAGE) algorithm is another explore-then-exploit algorithm, as are BTM and IF.
	As in RUCB and RCS this algorithm runs under the assumption that a Condorcet winner exists.
	The algorithm pseudo-code is presented in Figure~\ref{algo_SAVAGE}.
	In general, the algorithm compares pair of  arms until there exists a pair for which one of the arms beats the other by a wide margin, in which case the loser is removed from the working set.
	More precisely, in each round two arms, with the least amount of plays, are chosen, as seen in line~\ref{alg:savage:select_xy}.
	The pair of arms are compared and given a score, $u_{x_t, y_t}$ (Line~\ref{alg:savage:score_xy}).
	An arm $x$ that does not uphold the condition
	$$\sum_{y}\mathbbm{1}(u_{x,y} > 0.5) < K $$
	is eliminated from the working set.
	This condition is very similar to the condition in the RUCB algorithm.
	Although this algorithm seems like "more of the same", this algorithm eliminates arms from the working set (as seen in IF and BTM) according to the likelihood of the arm being the Condorcet winner, while RUCB and RCS only pick the most likely pair of arms to hold the Condorcet winner.
	This algorithm guarantees the following:
	\begin{theorem}\label{thm:SAVAGE}[From \cite{urvoy2013generic}]

		Running the SAVAGE algorithm with $|X|=K$, with an unknown infinite time horizon of $T>K$, the expected regret is bounded by $$R_P(T) = \mathcal{O} \left( K^2 log( T ) \sum\limits_{x\in X \backslash x^*} \frac{1}{\Delta_{x,x^*}^2}\right)$$

	\end{theorem}
	
\begin{figure}[h]
	\IncMargin{1em}
		\begin{algorithm}[H]
		\SetKwInput{KwInput}{Input}
		
		\KwInput{$X ,K=|X|, \alpha$}
			$ \mathcal{P} \leftarrow X $\\
			$ \forall x \in \mathcal{P}, n_x \leftarrow 0 $\\
		\BlankLine
		\While{$|\mathcal{P}|>1 \wedge t \leq T$}{			
			$x_t, y_t \leftarrow argmin_{x}{n_{x}}$\label{alg:savage:select_xy}\\
			Compare($x_t, y_t$)\\
			$n_{x_t} \leftarrow n_{x_t}+1$\\
			$n_{y_t} \leftarrow n_{y_t}+1$\\
			update($W$)\\
			$G = [g_{x_t, y_t}]\leftarrow \frac{w_{x,y}}{w_{x,y}+w_{y,x}} +\sqrt{\frac{\alpha}{w_{x,y}+w_{y,x}}}$\label{alg:savage:score_xy}\\
			$\mathcal{P}  \leftarrow \mathcal{P}\backslash \left\{x\mid 				\sum_{y}\mathbbm{1}(g_{x,y} > 0.5) < K)\right\}$\\
			$t \leftarrow t+1$
		}
		\Return{ $x_t $}
			\caption{SAVAGE}
		\end{algorithm}
		\caption{SAVAGE Algorithm}\label{algo_SAVAGE}
	\end{figure}
	\newpage
	
\subsection{Black-Box Algorithms}
	In \cite{ailon2014reducing}, Ailon et al. tried to tackle the dueling bandits problem (UBDB-version) using standard bandit algorithms.
	In their work, Ailon et al. use MAB algorithm black boxes, defined as Singleton Bandits Machine (SBM) - closed computational units with internal timer and memory.
	A SBM $S$ supports three operations: reset, advance, and feedback. The reset operation simply clears the SBM's state.
	The advance operation returns the next chosen arm, and feedback is used for simulating a feedback (the utility).
	\paragraph{Doubler}
	(Ailon et al., 2014) handles a large or possibly infinite set of arms $X$ for the infinite horizon case.
	The algorithms pseudo-code is presented in Figure~\ref{algo_Doubler}.
	This algorithm is best explained by thinking of a competition between two players; the first controls the choice of the left arm and the second player controls the right arm.
	The objective of each player is to win as many rounds as possible.
	This algorithm divides the time axis to exponentially growing epochs $p$ (first epoch is 2 rounds, second epoch is 4, third epoch is 8, and so forth).
	In each epoch, the left side plays according to a fixed stochastic strategy, which will be explained shortly, while the right one plays adaptively according to a strategy provided by a SBM black-box.
	In the beginning of a new epoch (line~\ref{line:new_epoch}) the distribution governing the left arm changes in a way that mimics the actions of the right arm in the previous epoch.
	More precisely, the left side plays according to a distribution defined by the multi-set $\mathcal{L}$ that holds all the plays the right side performed in the previous epoch.
	The algorithm guarantees the following:
	\begin{theorem}\label{thm:Doubler}[From \cite{ailon2014reducing}]

		Running the Doubler algorithm with $|X|=K$, with an infinite time horizon, the expected regret is bounded by $R_U(T) = \mathcal{O} \left(H log^2 T \right)$ for all $T>0$, where $H=\sum_{x\in X \backslash x^*}^K \Delta^{-1}_x$ and the regret as in \eqref{eq:utility_regret}, and $\Delta_x$ is the gap between arm $x$ and the optimal arm.

	\end{theorem}
	As can be seen in line~\ref{line:new_epoch} of Algorithm~\ref{algo_Doubler}, $\mathcal{S}$ is reset every time a new epoch begins, by removing all the information gained in the previous rounds. This procedure inflicts the extra $\log(T)$ factor to the accumulated regret.
	
	\begin{figure}[h]
	\IncMargin{1em}
		\begin{algorithm}[H]
		\SetKwInput{KwInput}{Input}
		
		\KwInput{$X$, $\mathcal{S}$}
		\BlankLine
		$p\leftarrow 1$\\
		$ \mathcal{L} \leftarrow X$\\
		\While{True}{
			reset($\mathcal{S}$) \label{line:new_epoch}\\
			\For{$t = 2^{p-1}$ \KwTo $2^{p}$}{
				choose $x_t$ uniformly from $\mathcal{L}$\\
				$y_t \leftarrow advance(\mathcal{S})$\\
				compare $(x_t , y_t )$, observe choice $b_t$\\
				feedback $\left(\mathcal{S},b_t\right)$ \label{alg:dblr:feedback}\\
			}
			$\mathcal{L} \leftarrow$the multi-set of arms played as y t in the last for-loop.\\
		$p \rightarrow p+1$
	}
			\caption{Doubler}
		\end{algorithm}
		\caption{Doubler Algorithm}\label{algo_Doubler}
	\end{figure}
\newpage
\paragraph{Sparring}
	(Ailon et al., 2014)\cite{ailon2014reducing} The algorithm pseudo-code is presented in Figure~\ref{algo_Sparring}.
	The idea for the Sparring algorithm comes from analysis of an adversarial version of UDBD, in which it can be easily shown that the resulting expected regret of Sparring is at most a constant times the regret of the two SBM black-boxes that implement an algorithm for adversarial MAB (i.e., $EXP3$).  We do not elaborate on these arguments and on adversarial
	games in these thesis.
		
		\begin{figure}[h]
	\IncMargin{1em}
		\begin{algorithm}[H]
		\SetKwInput{KwInput}{Input}
		\KwInput{$X, \mathcal{L} = \{x_1\}$, $ \mathcal{S}_R, \mathcal{S}_L$}
		\BlankLine
		$t\leftarrow 1 $\\
		\While{True}{
			$x_t\leftarrow$ advance($\mathcal{S}_L$),
			$y_t\leftarrow$ advance($\mathcal{S}_R$)\\
			compare($x_t, y_t$)\\
			observe $b_t$\\
			feedback($\mathcal{S}_L$, $\mathbbm{1}_{b_t=0}$)\\
			feedback($\mathcal{S}_R$, $\mathbbm{1}_{b_t=1}$)\\
			$r\leftarrow t+1 $
		}
			\caption{Sparring}
		\end{algorithm}
		\caption{Sparring Algorithm}\label{algo_Sparring}
	\end{figure}
	
	\begin{conjecture}
 		In the paper Ailon et al., 2014 conjectured that the utility based regret of the Sparring algorithm is bounded by the combined regret of the SBM black-boxes, with a possibility of a small  additive overhead.
 	\end{conjecture}
 	
 	Proving the conjecture appears to be tricky due to the fact
that the left (resp. right) SBM black-box does not see a stochastic environment, because its feedback depends on non-stochastic actions of the two SBM black-boxes. 

\chapter{Our Approach}
We start with two algorithms that try to fix a major shortcoming of the Doubler algorithm. In every epoch, the Doubler algorithm resets the SBM black-box, meaning that it will need to re-learn the environment thus incurring an extra $\log(T)$ factor on the cumulative regret.
	We present two algorithms, Improved Doubler and Balanced Doubler, that tackle this shortcoming.
	
	\section{Improved Doubler}
	The Improved Doubler algorithm's pseudo-code is presented in Figure~\ref{algo_adv_doubler}.
	Instead of initializing the right arm's SBM at each epoch and losing all the information that was learnt, we will store an invariant of this data.
	Let $D_p$ denote the distribution of the left arm in the $p$'th epoch.
	Had we known $f_p := E_{x\in D_p} \mu_{x}/2$, we could have used $b_t + f_p$ as feedback on the right side.
	This would have offset the change in the expected utility of the left arm, from epoch to epoch.
	As a result, we would not have had to reset the SBM at each epoch.
	We fix this by changing line~\ref{alg:dblr:feedback} in the Doubler algorithm to feedback($\mathcal{S}, b_t + f_p $).
	Assuming the link function $\phi = \phi_{lin}$, meaning $\phi(u,v) = \frac{1+v-u}{2}$, it is easy to see that $E(b_t + f_p )$ is exactly $E(\frac{u(y_t)+1}{2})$, meaning we have decoupled the arms' utilities.
	Of course we do not know $f_p$, and can only estimate it.
	Luckily, an unbiased estimate of $f_p$, $\hat{f}_p$ can be obtained using a simple inductive formula.
	First we notice $E[f_p]$ is defined by 
	\begin{equation}
	E[f_{p+1}] = \frac{1}{|T_p|}\sum\limits_{t\in T_p} \mu_{y_t}/2
	\end{equation}
	thus:
	$$
	E[f_{p+1} ]= 
		\frac{1}{|T_p|}\sum\limits_{t\in T_p} E[b_t - \frac{1-\mu_{x_t}}{2}] = \frac{1}{|T_p|}\sum\limits_{t\in T_p} E[b_t + f_{p-1}-\frac{1}{2}] = E[f_{p-1} ]+E[b_t -\frac{1}{2}]
	$$
	The estimate for $f_p$ is given by a recursive formula:
	\begin{equation}
	\hat{f}_p=\hat{f}_{p-1} + \frac{\sum\limits_{t\in T_p} (2b_t -1)}{|T_{p-1}|}
	\end{equation}
	We wish to bound $\hat{f}_p$:
	
	$$\hat{f}_{p+1}=\hat{f}_{p} + 
		\frac{\sum\limits_{t\in T_p} (2b_t -1)}{|T_{p}|} = 
		\hat{f}_{p} + 
		\frac{\sum\limits_{t\in T_p} (2b_t -1)}{2^{p}} = 
		\hat{f}_{p} + 
		\frac{\sum\limits_{t\in T_i} 2b_t}{2^{p}}-1 =$$
		$$
		\hat{f}_{p} + B_p -1 =  
		\hat{f}_{p-1} +B_{p-1} -1+ B_p -1 = ... =  
		\hat{f}_{0} + \sum\limits_{j=1}^p B_{j}+p\ ,$$
	where $ B_{p} = \frac{\sum\limits_{t\in T_p} 2b_t}{2^{p}}$ is the average of $b_t$ in epoch $p$, which is bound by one, meaning 
	$\sum\limits_{j=1}^p B_{j}$ is bound by $p$ thus bounding $\hat{f}$ by $\log(T)$.
	This is a very rough bound of $\hat{f}_p$, with it we cannot prove that the total regret achieve a better bound than $\mathcal{O}(\log^2(T))$.
	In our experiments we have witnessed that $\hat{f}_p$ is far smaller than the rough bound. A high probability bound of $\mathcal{O}(1)$ would prove that the total regret is bound by $\mathcal{O}(\log(T))$.
	We have yet to prove this and leave this for further research.
	
\begin{figure}[h]
	\IncMargin{1em}
		\begin{algorithm}[H]
		\SetKwInput{KwInput}{Input}
		\KwInput{$X, \hat{f_0} = 0$, $\mathcal{S}$}
		\BlankLine
		$\mathcal{L} \leftarrow X$\\
		$p\leftarrow 1$\\
\While{True}{
	\For{$t = 2^{p-1}$ \KwTo $2^{p}$}{\label{forins}
		choose $x_t$ uniformly from $\mathcal{L}$\\
		$y_t \leftarrow advance(\mathcal{S})$\\
		compare $(x_t , y_t )$, observe choice $b_t$\\
		feedback $\left(\mathcal{S},b_t + \hat{f}_{p-1}\right)$\\
	}
	$\mathcal{L} \leftarrow$the multi-set of arms played as y t in the last for-loop.\\
	$\hat{f_{p}} \leftarrow \hat{f}_{p} + \sum_{s\in T_p} b_s /2^{p-1} - 1/2$\\
	$p \rightarrow p+1$
}
			\caption{Improved Doubler}
		\end{algorithm}
		\caption{Improved Doubler Algorithm}\label{algo_adv_doubler}
	\end{figure}
\newpage
	\section{Balanced Doubler}
		Another improvement of the Doubler algorithm is the Balanced Doubler algorithm. In this algorithm we assume we know $min_{x\in X \backslash x^*}(\Delta_x)$.
		Instead of resetting the SBM on each epoch as in the Doubler algorithm, we will use an estimator as defined in the Balanced Doubler.
In order to cancel the offset in the estimator we will use average of averages:
\begin{equation}
\tilde{\mu}(y) = E\left[ \frac{1}{P}\sum\limits_{p=1}^P \frac{1}{n_{y,p}} \sum\limits_{t\in T_{y,p}} b_t\right]
\end{equation}
where $n_{y,p}$ is the number of times arm $y$ was chosen in epoch $p$.
We will prove that this estimate is an unbiased estimate of $\mu_{y}$ up to a constant that is invariant of $y$:
\begin{equation*}
E\left[ \frac{1}{P}\sum\limits_{p=1}^P \frac{1}{n_{y,p}} \sum\limits_{t\in T} \frac{\mu_{y}-\mu_{x_t}+1}{2}\right]= \mu_{y} +\frac{1}{2} +E\left[ \frac{1}{P}\sum\limits_{p=1}^P \frac{1}{n_{y,p}} \sum\limits_{t\in T_{y,p}} \frac{\mu_{x_t}}{2}\right]
\end{equation*}
In short: 
\begin{equation}
\tilde{\mu}(y)  = \mu_{y} + C(P) \ .
\end{equation}
We will now show that in order to achieve $\mathcal{O}(\log T)$ regret bound we need a warm up phase.
We define $s$ to be the number of times each arm is played in the warm-up phase.
	Due to the fact that $\Delta_{x} \in [0,1]$, the total regret can be bound by:
	\begin{equation*}
	R=R_{warm-up}+R_{rest} \leq s \cdot p K + 
	\sum_{t=1}^{\infty}Pr\left(\tilde{\mu}(x_t)>\tilde{\mu}(x^*)\;|\;{x_t\in X \backslash x^*} \right)
	\end{equation*}
	
	We wish to bound $Pr(\tilde{\mu}(x)>\tilde{\mu}(x^*))$. 
	In the worst case scenario (standard deviation-wise) $n_{x,p} \equiv s$, meaning that arm $x$ was chosen in each epoch only $s$ times, $s\cdot p$ in total.
	The probability of choosing the wrong arm can be bound by:
	\begin{equation}\label{regret_bound}
	Pr(\tilde{\mu}(x)>\tilde{\mu}(x^*)) < 
	Pr(\tilde{\mu}(x) > \mu_{x} + \frac{\Delta_x}{2})+ 
	Pr(\tilde{\mu}(x^*)< \mu_{x^*} - \frac{\Delta_x}{2})
	\end{equation}
	
	Meaning that at epoch $p$ suboptimal arm $x$ was played $s\cdot P$ times and so we can bound the probability of that arm being played again by:
	\begin{equation}
	Pr(\tilde{\mu}(x) > \mu_{x} + \frac{\Delta_x}{2})+ 
	Pr(\tilde{\mu}(x^*)< \mu_{x^*} - \frac{\Delta_x}{2}) \leq \sum\limits_{p=1}^P 2 e^{-\frac{\Delta^2_x}{2}s\cdot p}
	\end{equation}
	And the regret incurred by arm $x$ can be bound by:
	\begin{equation}
	R_x < s\cdot P\Delta_x+2\Delta_x\sum\limits_{p=1}^P \left( 2^p e^{-\frac{\Delta^2_x}{2}s\cdot p} \right) = s\cdot P\Delta_x+2\Delta_x\sum\limits_{p=1}^P \left( 2 e^{-\frac{\Delta^2_x}{2}s} \right)^p
	\end{equation}
	By using the sum of geometric series we can bound the regret by:
	\begin{equation}
	R_x < s\cdot P\Delta_x+2\Delta_x \frac{2 e^{-\frac{\Delta^2_x}{2}s} - (2 e^{-\frac{\Delta^2_x}{2}s})^P}{1-2 e^{-\frac{\Delta^2_x}{2}s}}
	\end{equation}
	We wish to find $s$ that will yield the minimum regret.
	We will split the calculation of $s$ :	
	$$s < \frac{2\log(2)}{\Delta^2_x}$$
	Assuming the above, we get:
	
	\begin{equation}
		R_x < s\cdot P\Delta_x+2\Delta_x \frac{2 e^{-\frac{\Delta^2_x}{2}s} - (2 e^{-\frac{\Delta^2_x}{2}s})^P}{1-2 e^{-\frac{\Delta^2_x}{2}s}} \sim s\cdot P\Delta_x - (2 e^{-\frac{\Delta^2_x}{2}s})^P
	\end{equation}	
	We will derive and get
	\begin{align*}
	& P\Delta_x + \Delta_x\frac{P\Delta^2_x}{2}(2 e^{-\frac{\Delta^2_x}{2}s})^P = 0
	\\
	& 1 +\frac{\Delta^2_x}{2}(2 e^{-\frac{\Delta^2_x}{2}s})^P = 0
	\\
	& -\frac{2}{\Delta^2_x} = 2^P e^{-\frac{P\Delta^2_x}{2}s}
	\\
	& \log(\frac{\Delta^2_x}{2}) = -\frac{P\Delta^2_x}{2}s + P\log(2)
	\\
	& \log(\frac{\Delta^2_x}{2}) - P\log(2) = -\frac{P\Delta^2_x}{2}s
	\\
	& \frac{P\Delta^2_x}{2}s = P\log(2) - \log(\frac{\Delta^2_x}{2})
	\\
	& s = \frac{2\log(2)}{\Delta^2_x} - \frac{\log(\frac{\Delta^2_x}{2})}{P\Delta^2_x}
	\end{align*}
	Taking  $P\rightarrow \infty$ gives us $s = \frac{2\log(2)}{\Delta^2_x}$.
	In the second case:
	$$s > \frac{2\log(2)}{\Delta^2_x}$$
	Assuming the above we get:
	\begin{equation}
		R_x < s\cdot P\Delta_x+2\Delta_x \frac{2 e^{-\frac{\Delta^2_x}{2}s} - (2 e^{-\frac{\Delta^2_x}{2}s})^P}{1-2 e^{-\frac{\Delta^2_x}{2}s}} \sim s\cdot P\Delta_x + 2\Delta_x \frac{2 e^{-\frac{\Delta^2_x}{2}s} }{1-2 e^{-\frac{\Delta^2_x}{2}s}}
	\end{equation}
	And from here we can see that the minimum of this function (assuming $s > \frac{2\log(2)}{\Delta^2_x}$) is $\frac{2\log(2)}{\Delta^2_x}$.

	Now that we have the number of rounds each arm is played in the warm start phase we calculate the regret.
	For a single arm $x$:
	\begin{equation}
	R_x < \log(4)P\frac{1}{\Delta_x} +2\Delta_x\sum\limits_{p=1}^P \left( 2 e^{-\frac{\Delta^2_x}{2}\frac{2\log(2)}{\Delta^2_x}} \right)^p = P(\frac{\log(4)}{\Delta_x}+2\Delta_x)
	\end{equation}
	
	And for all the arms we get:
	\begin{equation}
	R(P) = P \sum\limits_{x\in X\backslash x^*} (\frac{\log(4)}{\Delta_x}+2\Delta_x)
	\end{equation}
	And where $P$ is of $\mathcal{O}(\log(T))$ we get the regret is of $\mathcal{O}(\log(T))$.
	And we conclude with the following theorem:
	\begin{theorem}
	Running the Balanced Doubler algorithm with $|X|=K$, with an infinite time horizon, the expected regret is bounded by $R(T) = \mathcal{O} \left(log T \right)$ for all $T>0$.
	
	\end{theorem}
	
	 
		
		\begin{figure}[h]
	\IncMargin{1em}
		\begin{algorithm}[H]
		\SetKwInput{KwInput}{Input}
		\KwInput{$X$, $\mathcal{S}$}
		\BlankLine
		$\mathcal{L} \leftarrow X$\\
		$p\leftarrow 1$\\
		\While{True}{
		Play each arm $\frac{k}{min_{x\in X \backslash x^*}(\Delta_x)}$ times against all other arms\\
		\For{$t = 2^{p-1}$ \KwTo $2^{p}$}{
		choose $x_t$ uniformly from $\mathcal{L}$\\
		$y_t \leftarrow argmax_i
			\left(
				\frac{1}{p}\sum\limits_{j=1}^{p}\frac{1}{n_{i,j}}
						   \sum\limits_{t\in T_{i,j}} b_t
				+
				\sqrt{
					\frac{\alpha}{p^2}\cdot\log(t)
					\cdot(
						\sum\limits_{j=1}^{p}\frac{1}{n_{i,j}}
						)
					}
			\right)$\\
		compare $(x_t, y_t)$\\
		observe choice $b_t$\\
		}
	$\mathcal{L} \leftarrow$ the multi-set of arms played as $y_t$ in the last for-loop.\\
	$p \rightarrow p+1$
}
			\caption{Balanced Doubler}
		\end{algorithm}
		\caption{Balanced Doubler}\label{algo_balanced_doubler}
	\end{figure}
	\newpage
		\section{Sparring with Thompson Sampling}
	The main drawback of the Balanced Doubler algorithm is the warm-start phase.
	One of the benefits of the Thompson Sampling algorithm is that it does not need any warm start phase.
	As mentioned in section 2.1.2, the Thompson Sampling algorithm samples the random variables $\Theta_x, x\in X$ with Beta distribution and picks the arm with the highest value.
	Until an arm $x$ is played, the random variable holds a uniform distribution, this roughly explains why the warm-start phase is not needed.
		The Sparring with Thompson Sampling algorithm pseudo-code is presented in Figure~\ref{algo_SparringTS}.
		The algorithm maintains the number of times each arm won and lost a comparison.
		In each round the algorithm samples all the random variables for each of the SBMs: $\Theta_{x,R}, \Theta_{x,L}$ (for each arm $x \in X$) with Beta distribution. 
		The Beta distribution is defined by the hyper-parameters $Success_{L,x}$, $Fails_{L,x}$ for the left set of random variables and $Success_{R,x}, Fails_{R,x}$ for the right set (lines~\ref{alg:sts:Theta_L}, \ref{alg:sts:Theta_R}).
		After the arms are compared and the outcome is observed all the relevant hyper-parameters are updated with the results.
		
	\begin{figure}[h]
	\IncMargin{1em}
		\begin{algorithm}[H]
		\SetKwInput{KwInput}{Input}
		
		\KwInput{$X$, $\mathcal{S}_R, \mathcal{S}_L$}
		\BlankLine
		$t\leftarrow 1$\\
\While{True}{
	$\forall x \in X \Theta_{L,x} \sim Beta(Success_{L,x}+1, Fails_{L,x}+1)$ \label{alg:sts:Theta_L}\\
	$\forall x \in X \Theta_{R,x} \sim Beta(Success_{R,x}+1, Fails_{R,x}+1)$ \label{alg:sts:Theta_R}\\
	$x_t\leftarrow argmax_{x\in X}(\Theta_{L,x})$\\
	$y_t\leftarrow argmax_{x\in X}(\Theta_{R,x})$\\
	compare($x_t, y_t$)\\
	observe $b_t$\\
	$Success_{L,x_t} \leftarrow Success_{L,x_t} + \mathbbm{1}_{b_t=0}$\\
	$Fails_{L,x_t} \leftarrow Fails_{L,x_t} + \mathbbm{1}_{b_t=1}$\\
	$Success_{R,y_t} \leftarrow Success_{R,y_t} + \mathbbm{1}_{b_t=1}$\\
	$Fails_{R,y_t} \leftarrow Fails_{R,y_t} + \mathbbm{1}_{b_t=0}$\\	
	$t\rightarrow t+1$\\
}
			\caption{Sparring Thompson Sampling}
		\end{algorithm}
		\caption{Sparring Thompson Sampling Algorithm}\label{algo_SparringTS}
	\end{figure}
	
	\section{Sparring with Thompson Sampling Turbo}
	From the experiments conducted on the Sparring with Thompson Sampling algorithm we witnessed that in the initial phases the algorithm explores more than needed.
	In order to give the algorithm a more "aggressive behaviour" in the initial rounds and keep the exploration of the regular Thompson Sampling in advanced rounds, we added a time dependent factor of $\log(1+t\cdot 2^{-t/k^2}))$ to the Fails hyper parameter as seen in lines \ref{alg:stst:Theta_L} and \ref{alg:stst:Theta_R} in the Sparring with Thompson Sampling Turbo algorithm.
	By increasing the effective number of fails the arms with successes "stand out". We do not wish to keep this "aggressive behaviour" for the advanced rounds and hence $2^{-t/k^2}$ in $\log(1+t\cdot 2^{-t/k^2}))$ in order to remove this factor from advanced rounds.
	The Sparring with Thompson Sampling algorithms pseudo-code is presented in Figure~\ref{alg:SparringTS_turbo}.
	 
	\begin{figure}[h]
	\IncMargin{1em}
		\begin{algorithm}[H]
		\SetKwInput{KwInput}{Input}
		
		\KwInput{$X$, $\mathcal{S}_R, \mathcal{S}_L$}
		\BlankLine
		$t\leftarrow 1$\\
\While{True}{
	$\forall x \in X \Theta_{L,x} \sim Beta(Success_{L,x}+1, Fails_{L,x}+1+\log(1+t\cdot 2^{-t/k^2}))$ \label{alg:stst:Theta_L}\\
	$\forall x \in X \Theta_{R,x} \sim Beta(Success_{R,x}+1, Fails_{R,x}+1+\log(1+t\cdot 2^{-t/k^2}))$ \label{alg:stst:Theta_R}\\
	$x_t\leftarrow argmax_{x\in X}(\Theta_{L,x})$\\
	$y_t\leftarrow argmax_{x\in X}(\Theta_{R,x})$\\
	compare($x_t, y_t$)\\
	observe $b_t$\\
	%self.fails[arm]+1 + math.log(1+t*pow(2, -t/(pow(self.n_arms, 2))))
	$Success_{L,x_t} \leftarrow Success_{L,x_t} + \mathbbm{1}_{b_t=0}$\\
	$Fails_{L,x_t} \leftarrow Fails_{L,x_t} + \mathbbm{1}_{b_t=1}$\\
	$Success_{R,y_t} \leftarrow Success_{R,y_t} + \mathbbm{1}_{b_t=1}$\\
	$Fails_{R,y_t} \leftarrow Fails_{R,y_t} + \mathbbm{1}_{b_t=0}$\\	
	$t\rightarrow t+1$\\
}
			\caption{Sparring Thompson Sampling Turbo}
		\end{algorithm}
		\caption{Sparring Thompson Sampling Algorithm Turbo}\label{alg:SparringTS_turbo}
	\end{figure}

	\section{Unforgetful Thompson Sampling Doubler}
		Another improvement of the Doubler algorithm is the Unforgetful Thompson Sampling Doubler algorithm. Similarly to the Improved Doubler algorithm, we do not reset the SBM in each epoch and use the Thompson Sampling method as the SBM black-box.
		The idea for this algorithm came while trying to prove the logarithmic bound of the Improved Doubler algorithm.
		This approach outperformed all the other algorithms apart from the Thompson Sampling Sparring.
		The algorithm's pseudo-code is presented in Figure~\ref{algo_uts_doubler}.

		\begin{figure}[h]
	\IncMargin{1em}
		\begin{algorithm}[H]
		\SetKwInput{KwInput}{Input}
		\KwInput{$X$, $\mathcal{S}$}
		\BlankLine
		$\mathcal{L} \leftarrow X$\\
		$p\leftarrow 1$\\
		\While{True}{
			\For{$t = 2^{p-1}$ \KwTo $2^{p}$}{
				choose $x_t$ uniformly from $\mathcal{L}$\\
				$\forall x \in X \Theta_{R,x} \sim Beta(Success_{R,x}+1, Fails_{R,x}+1)$\\
				$y_t\leftarrow argmax_{x \in X}(\Theta_{R,x})$)\\
				compare $(x_t , y_t )$\\
				observe choice $b_t$\\
				$Success_{R,y_t} \leftarrow Success_{R,y_t} + \mathbbm{1}_{b_t=1}$\\
				$Fails_{R,y_t} \leftarrow Fails_{R,y_t} + \mathbbm{1}_{b_t=0}$\\	
			}
			$\mathcal{L} \leftarrow$the multi-set of arms played as $y_t$ in the last for-loop.\\
			$p \rightarrow p+1$
}
			\caption{Unforgetful Thompson Sampling Doubler}
		\end{algorithm}
		\caption{Unforgetful Thompson Sampling Doubler}\label{algo_uts_doubler}
	\end{figure}

\chapter{Experiments}

To evaluate our algorithms, we apply them on random synthetic data, special test cases (will be explained shortly), and also to the problem of scorer evaluation from the field of information retrieval (IR) (Manning et al., 2008) \cite{manning2008introduction}.  
A scorer is a function that takes a user's search query and ranks the documents in a collection according to their relevance to that query. Scorer evaluation aims to determine which set of scorers performs best.

Given a set of K scorers, the problem of finding the best scorer can then be modelled as a K-armed Dueling Bandit problem, with each arm relating to a scorer.
This approach follows an experimental paradigm that has previously been used for assessing the performance of scorers \cite{radlinski2008does}.

\section{Real Data}
Our experiments is built on standard IR datasets, namely the LETOR MQ2007 and MSLR-WEB30K datasets (Liu et al., 2007).\footnote{These data sets can be retrieved from http://research.microsoft.com/en-us/um/beijing/projects/letor/}
MQ2007 holds about 65 million rows and MSLR-WEB30K holds 30K rows.
Each row in the data is a query-URL pair, where the number in the beginning of the row signifies the relevance of this result and each of the following pairs is a scorer-score pair.
Here is an example of a single row from MQ2007:
\begin{equation}\label{queri}
2 \text{ qid}:10032, 1:0.0565, 2:0.0000, 3:0.6667 ... 45:0.0000, 46:0.0769
\end{equation}
This data set holds 46 scorers, each relating to a ranking feature provided in the data set.
These scores are used in order to rank the relevance of a search query to a document. A good example of a feature is the number of times the query shows up in the body of the document. 
The MSLR-WEB30K is built in the same manner; the only difference is the number of scorers (136 in total).
We use two different settings, BPDP and UBDB; for each of these settings we use the data sets in a different manner.
\subsection{Converting LETOR data to a PBDB matrix}
We used the following method to convert the method to a PBDB matrix $P$ over the set of scorers.
The method is based on a combination of standard IR techniques for simulating noisy comparisons between lists. We show the method in the pseudo-code in Figure~\ref{data_converting}.
In order to set the preference matrix we need to compare between the scorers/arms; in our setting we use the Interleaved Comparison method (Radlinski et al., 2008) \cite{radlinski2008learning}.
The main idea behind the Interleaved Comparison is deciding which of the two given alternatives is preferred.
In particular, Radlinski proposed a method for presenting the results from two rankers, where a ranker is a function that when given a query, returns a list of elements (i.e., websites, documents, actions) ordered according to the ranker's functionality (i.e., number of times the query showed up in the text of the document).
The two rankers provide a rank, for instance ranker $A$ provides $(a_1,a_2,a_3)$ and ranker $B$ provides the rank $(b_1,b_2,b_3)$.
The interleaved rank is a "merge" of the ranks (i.e., $(a_1,b_1,a_2)$).
For each query an interleaved rank is presented to the user. If the user clicks $a_1$ from the given example we assume that ranker $A$ is preferred over ranker $B$.
In order to create a ranking several query-URL pairs are taken, with the same query but different URLs, ranked them according to the relevant scorers, and then the two ranks are interleaved into an interleaved rank.
Sets of rows with the same query and different URL's are sampled randomly, interleaved into an interleaved rank and clicks are generated probabilistically, according to the score of the relevant scorer for the sampled query. More precisely, for each document in the list, starting from the top of the list, a click is generated according the probability $P(\text{click}|\text{scorer's score for the pair})$, meaning the higher the scorer's score the higher the probability of the document being clicked.
In order to set $P = [p_{x,y}]$ each arm is compared to every other arm, using the interleaved comparison method, and the empirical probability is set.

\begin{figure}[h]
	\IncMargin{1em}
		\begin{algorithm}[H]
		\SetKwInput{KwInput}{Input}
		
		\KwInput{LETOR Data Set, $X, K=|X|$}
		$W =[w_{x,y}]$\\
		$P =[p_{x,y}]$
		\BlankLine
		$\mathcal{S} \leftarrow N$ sets of queries-url with the same query from data-set.\\
		\For{$t = 1$ \KwTo $80,000$}{
		\ForAll{$x\in X$}{
			\ForAll{$y\in X$}{
			$s_x\leftarrow$ set from $\mathcal{S}$ sorted according to scorer $x$\\
			$s_y\leftarrow$ set from $\mathcal{S}$ sorted according to scorer $y$\\	
			$I\leftarrow$ interleaved set of $s_x,s_y$.\\
			Generate a click.\\
			If a row from set $s_x$ was clicked $w_{x,y}\leftarrow w_{x,y}+1$\\
			If a row from set $s_y$ was clicked $w_{y,x}\leftarrow w_{y,x}+1$
			}
		}
		}
		$P =[p_{x,y}]\leftarrow \left[\frac{w_{x,y}}{w_{x,y}+w_{y,y}}\right]$

			\caption{Converting LETOR Data to the Preference Matrix}
		\end{algorithm}
		\caption{Converting LETOR Data Procedure}\label{data_converting}
	\end{figure}

In order to evaluate our algorithms in this setting we use the PBDB regret as defined in \eqref{eq:preference_regret}

\subsection{Converting LETOR data to UBDB utilities}

In the UBDB setting, in each round $t$, a query is sampled from the MQ2007 data-set, the unobserved utilities $u_t,v_t$ are set according to the chosen arms, and finally the observed feedback $b_t$ is defined by a Bernoulli trail with $ p = \frac{u_t -v_t+1}{2}$.

If we use the above example \eqref{queri} and set the chosen arms to $x_t = 1, y_t=3$ then $b_t \sim \textit{Bernoulli}\left(\frac{0.0565 -0.6667+1}{2}\right)$.
In order to evaluate our algorithm in this setting we use the UBDB regret as defined in \eqref{eq:utility_regret} and $\mu_x$, the average utility of arm $x$ (that is needed in order to calculate the regret) is set to be the average of scores in the data set for scorer $x$.

\section{Synthetic Data}
In order to evaluate how our algorithm reacts to special cases of input we use synthetic data that is predefined.
In our experiments we let our algorithm handle the following scenarios:
\begin{enumerate}
\item Linear Mean Utilities
\item Small Gap
\item Big Gap
\item Random Mean Utilities
\end{enumerate}
The first special case we evaluate is Linear Mean Utilities, for example with 5 arms $(\mu_1, \mu_2, \mu_3, \mu_4, \mu_5) = (0, 0.25, 0.5, 0.75, 1)$.
Two other special cases are when there is one mean utility that is notably higher than all the other arms, and the last case is where the mean utilities of all the arms a fairly close .
The last case we simulate is random mean utilities.
The regret is defined similarly to the real data experiments.

\newpage
\section{Results}
In this section we will be showing simulation results for PBDB and UBDB setting, with various algorithms and several values of K for $T = 2^{17}$ time steps. 
Each graph shows the empirical mean of the Cumulative Regret, over $M =200$ independent runs.
The mean of the Cumulative Regret for each time step is simply:
	\begin{equation}
	 \hat{R}(t) = \frac{1}{M}\sum_{m=1}^{M} R_m(t)
	\end{equation}
where $R_m(t)$ is cumulative regret as defined in \eqref{eq:utility_regret} and \eqref{eq:preference_regret}.
For the convenience of the reader we provide a table of all the simulations and relevant figures:

\begin{table}[h!]
\centering
\begin{tabular}{ |p{2cm}|p{2cm}|p{4cm}|  }

\hline
 \multicolumn{3}{|c|}{Simulation} \\
 \hline
   \hline
Figure & K & Test Case \\
\hline
  16     & 8  & Small Gap  \\
  \hline
  17     & 16  &  Small Gap  \\
  \hline
  18     & 46  &  Small Gap  \\
  \hline
  19     & 8  &  Linear Mean Utilities \\
  \hline
  20     & 16  & Linear Mean Utilities  \\
  \hline
  21     & 46  & Linear Mean Utilities  \\
  \hline
  22     & 8  &  Big Gap \\
  \hline
  23     & 16  &  Big Gap \\
  \hline
  24     & 46  &  Big Gap \\

  \hline
  25     & 8  &  Random Mean Utilities  \\
  \hline
  26     & 16  &  Random Mean Utilities  \\
  \hline
  27     & 46  &  Random Mean Utilities  \\
  \hline
  28     & 8  &  MQ2007 Data Set \\
  \hline
  29     & 16  & MQ2007 Data Set  \\
  \hline
  30     & 46  & MQ2007 Data Set  \\
  \hline
  31     & 46  & WEB30K Data Set  \\
  \hline
\end{tabular}
\end{table}
In some test cases, some algorithms are extremely out-performed by the other algorithms; in those cases we leave out the plot of the cumulative regret of those algorithms.
\newpage
\subsection{Small Gap}
In these experiments we have 3 arms whos mean utilities are $1, 0.995, 0.99$, where the rest of the arm's utilities are spread uniformly between $0.5$ and $1$. 
\begin{figure}[h!]
\centering
  \includegraphics[scale=0.8]{graphs/close_arm_8.png}
  \caption{Small Gap - 8 Arms}
\end{figure}
\newpage
\begin{figure}[h!]
\centering
  \includegraphics[scale=0.8]{graphs/close_arm_16.png}
  \caption{Small Gap - 16 Arms}
\end{figure}
\newpage
\begin{figure}[h!]
\centering
  \includegraphics[scale=0.8]{graphs/close_arm_46.png}
  \caption{Small Gap - 46 Arms}
\end{figure}
\newpage
\subsection{Linear Mean Utilities}
In these experiments we the mean utilities are spread uniformly between $1$ and $0$.
\begin{figure}[h!]
\centering
  \includegraphics[scale=0.8]{graphs/linear_arm_8.png}
  \caption{Linear Mean Utilities - 8 Arms}
\end{figure}
\newpage
\begin{figure}[h!]
\centering
  \includegraphics[scale=0.8]{graphs/linear_arm_16.png}
  \caption{Linear Mean Utilities - 16 Arms}
\end{figure}
\newpage
\begin{figure}[h!]
\centering
  \includegraphics[scale=0.8]{graphs/linear_arm_46.png}
  \caption{Linear Mean Utilities - 46 Arms}
\end{figure}
\newpage
\subsection{Big Gap}
In these experiments there is one arm with a mean utility of $1$ and the rest of the arms are spread uniformly between $0.5$ and $0$.
\begin{figure}[h!]
\centering
  \includegraphics[scale=0.8]{graphs/one_strong_arm_8.png}
  \caption{Big Gap - 8 Arms}
\end{figure}
\newpage
\begin{figure}[h!]
\centering
  \includegraphics[scale=0.8]{graphs/one_strong_arm_16.png}
  \caption{Big Gap - 16 Arms}
\end{figure}
\newpage
\begin{figure}[h!]
\centering
  \includegraphics[scale=0.8]{graphs/one_strong_arm_46.png}
  \caption{Big Gap - 46 Arms}
\end{figure}
\newpage
\subsection{Random Mean Utilities}
In these experiments all the mean utilities are random.

\begin{figure}[h!]
\centering
  \includegraphics[scale=0.8]{graphs/random_data_8.png}
  \caption{Random Mean Utilities - 8 Arms}
\end{figure}
\newpage
\begin{figure}[h!]
\centering
  \includegraphics[scale=0.8]{graphs/random_data_16.png}
  \caption{Random Mean Utilities - 16 Arms}
\end{figure}
\newpage
\begin{figure}[h!]
\centering
  \includegraphics[scale=0.8]{graphs/random_data_46.png}
  \caption{Random Mean Utilities - 46 Arms}
\end{figure}
\newpage
\subsection{MQ2007 Data Set}
In these experiments the utilities of each arm are extracted from the data set MQ2007. 
\begin{figure}[h!]
\centering
  \includegraphics[scale=0.8]{graphs/real_data_8.png}
  \caption{MQ2007 Data Set - 8 Arms}
\end{figure}
\newpage
\begin{figure}[h!]
\centering
  \includegraphics[scale=0.8]{graphs/real_data_16.png}
  \caption{MQ2007 Data Set - 16 Arms}
\end{figure}
\newpage
\begin{figure}[h!]
\centering
  \includegraphics[scale=0.8]{graphs/real_data_46.png}
  \caption{MQ2007 Data Set - 46 Arms}
\end{figure}
\newpage
\subsection{WEB30K Data Set}
In this experiment the $P$ acquired from the WEB30K data set as described in section 5.1.1.

\begin{figure}[h!]
\centering
  \includegraphics[scale=0.8]{graphs/preference_real_data_46.png}
  \caption{WEB30K Data Set - 46 Arms}
\end{figure}
\newpage
\section{Conclusions of Results}
In the following table we have summarized the best performing algorithm for each of the test cases:
\begin{table}[h!]
\centering
\begin{tabular}{ |p{1cm}|p{1cm}|p{4cm}|p{5cm}| }

\hline
 \multicolumn{4}{|c|}{Simulation} \\
 \hline
   \hline
Figure & K & Test Case & Best Performing Algorithm\\
\hline
  16     & 8  & Small Gap & Sparring Thompson Sampling\\
  \hline
  17     & 16  &  Small Gap & Sparring Thompson Sampling Turbo\\
  \hline
  18     & 46  &  Small Gap & Sparring Thompson Sampling Turbo\\
  \hline
  19     & 8  &  Linear Mean Utilities & Sparring Thompson Sampling\\
  \hline
  20     & 16  & Linear Mean Utilities & Sparring Thompson Sampling Turbo\\
  \hline
  21     & 46  & Linear Mean Utilities & Sparring Thompson Sampling Turbo \\
  \hline
  22     & 8  &  Big Gap & BTM \\
  \hline
  23     & 16  &  Big Gap & BTM \\
  \hline
  24     & 46  &  Big Gap & RUCB\\

  \hline
  25     & 8  &  Random Mean Utilities & Sparring Thompson Sampling Turbo \\
  \hline
  26     & 16  &  Random Mean Utilities & Sparring Thompson Sampling \\
  \hline
  27     & 46  &  Random Mean Utilities & Sparring Thompson Sampling Turbo \\
  \hline
  28     & 8  &  MQ2007 Data Set & Sparring Thompson Sampling Turbo\\
  \hline
  29     & 16  & MQ2007 Data Set & Sparring Thompson Sampling \\
  \hline
  30     & 46  & MQ2007 Data Set & Sparring Thompson Sampling Turbo \\
  \hline
  31     & 46  & WEB30K Data Set & Sparring Thompson Sampling Turbo \\
  \hline
\end{tabular}
\end{table}
From this table it is clear that the Sparring Thompson Sampling Turbo and Sparring Thompson Sampling algorithm are best performing in most cases apart from the case of Big Gap where the elimination algorithm out-perform the black-box algorithms.
\chapter{Conclusions and Future Work}
This thesis has covered a range of algorithms for the Dueling Bandits Problem.
We have shown in our work a novel approach for tackling the Dueling Bandits Problem.
We show promising empirical results, comparable to the existing algorithms designed for the Dueling Bandits Problem and far outperforming them.
In future work, we will consider two extensions to this research.
First, we dealt with choice in sets of size 2. Extending the algorithm to $N$ choices would be an interesting direction.
Second, we have yet to try more complex white boxes. Sparring Thompson Sampling Turbo shows great promise, and modifying the hyper parameters may apply better performing algorithms as well as tighter theoretical bounds.
%that are comparable to the existing bounds for the other algorithms.

\bibliography{general}
\RPtrue
\authorHebrew{���� ��������}

\titleHebrew{������� ������ �� ������� ���-���}

\supervisorHebrew{����� ���� ������� ����' ��� ����� ������� ����� �����.}

%\personalThanksHebrew{����� ������ ������ �������}

\financialThanksHebrew{
��� ���� �������  ���� ���� �� ���� �� ������ ������ ������ ���������.
}

\JewishDateHebrew{���� �"� ���"�}

\GregorianDateHebrew{������� \beginL2015\endL}

\maketitleHebrew

% 1000-2000 words
\abstractHebrew

��������� )BAM( stidnaB demrA itluM ������ ������� ����� �� ����� ����� ������, ��� ��� \beginL1985\endL, ���� snibboR dna iaL  ������ �� ����� ]6[. ������ ��� ����� �� ���� �����, ���� ������ �� ���� �����  �-BAM  , �� �� ������ ������ �������, ��� �� ��� ������.  ���������� BAM  ������� ����, ������ ����� ����, ����: ����� ����� ]91 ,01 ,1 ,61[, ����� �����]8 ,22 ,81[ ������� �����]41 ,31[. 

����� �- BAM,  ����  ����� ���� ������ ������, ��� �� ����� ���� ����� ������ ��� )������ �� ������(, ���� �� ���������, ����� ������. ��� ���� ����� ��������, ��� �� ���� �����, ����� ���� �� �� ������ ������� )������ ������ tidnab demra eno ( ������� ����� ����� ��� ����, ��� ����� ��� ����� ������ ���. ���� ���� ����� ������, ���� ��� ��� ��� ����� ��������� ��������� ����� �����. ������ �� ����� ���� ����� ������ ����� ����� �� ��� ���� ����� ��� �������. �����, ����� ���� ������ ��� ��� ��� �����, ���� �� ������ ������� ����� ��� ��� �� ��, ����� �� ���� ������ ����. ���� ����� ���� ���� ��� ����� ������� ����� ������ �� ������ ��������, �����, ����� �� �����. ����� ����� ���� ����� ��� �� �� �������� ��� ���  ����� ����� �����, ���� �� ��  �������� ���� ����� ����, ��� ���� ��� ���� ���� ���� �� ����� ����������. ���� ����� �� ����� ����� ���� ���� ������ ������ ������. �����, ����� ����� �����, ����� ���� ����� ��� ���, ��� ���� ������ ����, ��� ������, ���� �� �� ������ ����� ����� �� ���� ���� ��� ���� ����� ���� ����. ���� �������� �� ����� ������ �- BAM, ���� ������ ���� ��� ���� ������� ����, ��� ������ ���� ����� �� ������� �����, �� �� ������ ����� ����� ����� ����, ���� ����� ������ ����, ��� ��� ������ ������ �������, ����� ���� ����� �����. ����� ��, ��� ����� �����, ����� ammeliD noitatiolpxE noitarolpxE.

������  ������ ���� ������ �- BAM. ���� ������ ������� ����, ��� ������ �� ���� ����� ���� ���� ����� �������, ���� ���� ����� ��� ��� �����, ������: "���� ���� 1 ����� ����� �� 7.0". ��� �� ���, ���� ����� ���� ��� ���� �� ���� �������, ������: ����� �����, ���� ����� ���� ���� ���� ��� �����, ��� ��� �������:  "���� ���� 1 ���� ���� ����� ���� 2". 

���� ����� ����, ����� ������, ��� ���� ����� ���� ���� ����� ���� ����� �� ���� �� ����.  ������ ���, ���� ������ � -gninraeL ecnereferP , ��� ���� ���. ����� �- gninraeL ecnereferP , ����� ���� �����, ��� �������, ���� ���, ������� �� ������� �� �������. 
����� �- stidnaB gnileuD ������ ����� ������� ����, ����� �� ����� ���� ��� �� ��� �����. ���� �� ����� ����� ���, ��� ���� ����� ���� ���� �� ��� ����� ������� ������ �� ����� �������
�������� �� �� �������. 

����� ����� ������ �����, ���� ��� ������� ������ �� ������ ����� ����� �� �- stidnaB gnileuD. ���� �����, ���� ���� ��� ������ �� ������ �������� ����� �������. ���� ����� ���� ����� �� ��� �����, ��� ����� ����� �������� ���������, ���� ����� ���� ���� �������� ���. ���� �� ����� ��, ����� ��� ������ ���� ����� ������, ���� ���� ������ ���� �����, ��� ������ �� ������ ��� �� ��� ���������. �����, ����� ������ ������ ���� ����� �� "��� ����� �� ������ ������� �� ��� ������ �����". ����� �����, ������ ����� ������ ��������, ���� ���� ����� ���� ����� �� ������ ����� ������. ������ ���� ������� ���� �� �������,  ��� ������ ����� �� ������, ������ �� ������ ���� ����� ���� ����� ������� �� ������. ����� ������ ����� ������: ����� ���� ��� ������ �� �����, ����� ������� ����� ��, ���� �����. �� ��� ����� �� ���� �������, �� ������ ����� ���� ������ �� �� ������, ����� ���� �� ����� ���� ���������, ����� ����� �� ���� ������� ���� �����, ��� �����. ��� ������� ����� ��� ������� ���������, ��� ����� ����� ������ �� �������, �� ���� ����� ���� ������. ����� ��, ���� "gnitseT B/A", ���� �� ���� ���� ���� ������ �- BAM. ����� �����: ���� �� ������ ����� ��������� �� �- BAM ? ��� ���� ������ �� ��� ���� ������ �� ����� ���� ����� ������ �� �������. ������: ������ �����, ���� ���� ����� �������. ����� ������� ���� ���� �� ��������� �� �� �������, ��� ���� ���� ���� ����� ����� ��� �� �� ������ �� ���� ��� ���� ���������, ��� ���� ��� ��� �� ��� ��������� ����� �����. ����� ��� �������� ����� �- BAM  �������� ��� ���� ������ ��� ��� ����, ����� �� ����� ���� ���� ������ �� ������ ������ �� �� �������. 

����� �� ��� ������� ���� ����� ������, ����� ���� "1"  ����� ���� ���� ����� "2" ������, ��������� stidnaB gnileuD  ������ ���� ����� ����� �������� �� ���� �����. �����, ����� ����� �������, ��� ����� ������ ������ ��� ������ ���������, ��� ����� ����� ����� �� ���� ������ �����. 
����� ���� �- BAM ��������� ������ �� ���� ���� �����, ������,  ���� ��� ����� ����� ��� ��� ������ ����� ���� �������, ���� ���� ������. ���� �� ����� ����� �� ����� ����� ��������. ���� ���, ��� ������ ���� ����������� �������� �� ���� "������� ������" ��� ����� �����]4[. ���� ���� ��� ������: ���� 2, ��� ������ �� ����� �- BAM, ����������� ��������� ����� ������ �- stidnaB gnileuD . ���� 3, ��� ������ �� ����������� �������� ����� ���� �������� �� ���� �"������� �������" ����� ������. ���� 4, ��� ���� ����������� relbuoD gnilpmaS nospmohT luftegrof ,obruT gnilpmaS nospmohT htiw gnirraps ,gnilpmaS nospmohT htiw gnirrapS ,relbuoD decnalaB ,relbuoD devorpmI. ���� 5, ��� ������ �� ��������,  ��� ������ ��� �������� �� ����������� ������� ���� 3 ���� ����������� ������� ���� 4. 
�������� ������ ������� ������ �� ����� ������� ���� ����� �� �����������, ���� �� ������� )teS ataD( ���� ����� ������� �� ROTEL - tfosorciM . ���� �� ���� ������ ����� �� ������� ��� ����� ����� ������ �������� ������ �����. ����� ����� �������, ������� ��� ������ ������, ��� ����� ��� ������ ��� �������� �� ����������� ���� ���� �������� �� ����������� ������, ��� ������ ������, ���� ����� ������� ����: ���� ������ ��� ���� ����� ����� ����, ���� ��� ���� ����� ����� ���� ���� ���� ������� ��� �� ������ ������. �������� ���, ����� �� ������� �� ����������� ���� ����� ����� �������,�� �� ��� �������� �� ���� ���� ������� ������ �������. ��� ������ �� ����� ��������� �- obruT gnilpmaS nosmohT htiw gnirrapS, ����� ������ ����� ���� ����� �- stidnaB gnileuD, ����� ����� gnitseT B/A ������� ������ ���� ������� ����� ��.


\finishedHebrew


\RPfalse
\end{document}
